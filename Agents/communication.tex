\section{Communication in Agent based Systems}

\index{agent-based system!communication}
The agents in a multi-agent system needs to interact with each other in order to realize the system functions. As in other 
distributed systems, the agents do not have any shared memory. Thus, the information possessed by an agent is strictly
private to itself. The agents can share information only by explicitly exchanging messages. 
%
For example, consider the case when a travel agent collaborates with several service providers to create a travel plan.
The proposed itinerary is private to the travel agent, and the information about the services offered are private to the 
service agents. The agents need to share the information explicitly to realize the travel plan.

The underlying network in a multi-agent systems is generally a IP network and, at the lowest level, the agents talk to 
each other with TCP, UDP or HTTP protocols. In cyber-physical systems, agents may need to operate on unreliable networks,
and specific techniques are deployed to account for the unreliability. In the rest of the chapter, we assume that the 
agents communicate over a reliable transport mechanism.
%
However, the transmission and reception of a byte-stream in a reliable manner is not sufficient for effective communication
between two agents. Since the agents are independently designed and are autonomous, they can transmit arbitrary byte-streams,
the meaning of which may not be comprehensible to the recipient agent. It is analogous to the fact that, in the human society,
we need to speak the same language and use the same meanings of the terms unambiguously to be understood. Thus, there needs 
to be an application-level protocol guiding the message structure and contents for the agents to unambiguously interpret the 
messages. Another aspect of the communication is the pragmatics, i.e. how an agent should respond to a message. For example, 
the social behavior in human society demands that, in response to a query ``can you tell me the time'', a person responds with 
something like``it is half past eight'' and not ``yes'', though the latter is also a logically valid response! 

\index{FIPA} \index{Foundation of Intelligent Physical Agents|see {FIPA}}
\index{KQML} \index{ARCOL}
While the agent technology had been maturing from research to industrial applications towards the later half of the 1990s, 
Foundation of Physical Agents (FIPA), a non-profit consortium of industries, took up the initiative of standardizing agent 
communications and interactions. The resulting specifications were based on earlier research efforts, primarily KQML~\citep{Finin:1994} 
and ARCOL (developed by French Telecom). Latest version of specifications from FIPA have been adopted by IEEE in 
2005. Presently, the standards are not being maintained, though they remain the guiding principles behind agent communication. 

Strictly speaking, the communication and interaction protocols conflict with the autonomy of the agents~\citep{Chopra:2013}. 
In order to comply with the standards, they cannot compose and transmit any arbitrary messages at any given context. An agent 
autonomy is also restricted when an agent needs to follow some social ethics, e.g. that it cannot retract from an earlier 
commitment communicated through a message in context of an interaction.

\subsection{Agent Communication Protocols}

\index{FIPA-ACL} \index{agent communication language} \index{speech-act theory}
\index{communicative act theory|see {speech-act theory}} \index{performative}
The FIPA model of agent communication is based on {\em speech-act theory}~\citep{Austin:1975}, which suggests that every 
communication has a purpose behind it, and is a part of a plan to achieve something. Though the theory had originally been 
developed for establishing a framework for human conversation (and hence the name), it has been found to have general 
application in non-verbal communication among humans as well as automated agents. Thus, some authors prefer to call it by 
a generic {\em communicative act theory}. While the 
purpose is often implicit (and hence ambiguous) in human speech, it is made explicit in agent communication language. For example,
instead of saying ``it is raining'', we can say ``I {\it inform} you that it is raining'' to make the intention of my speech
explicit. The explicit declaration of the purpose of communication is known as a {\em performative}. A performative can be 
classified into a few broad classes as shown in figure~\ref{fig:agents:performatives}. 

\begin{figure}[!htbp]
	\centering
	\epsfig{figure="Agents/performatives.eps",width=\linewidth}
	\caption{Classes of performatives}
	\label{fig:agents:performatives}
\end{figure}

\index{performative|textbf}
\begin{definition} [performative]
	A {\em performative} is an explicit declaration of the purpose of communication 
\end{definition}

\begin{figure}[!htbp]
	\centering
	\epsfig{figure="Agents/comm-model.eps",width=0.6\linewidth}
	\caption{FIPA Agent Communication Model}
	\label{fig:agents:comm-model}
\end{figure}

\index{performative} \index{communicative act|see {performative}}
The FIPA model of agent communication is shown in figure~\ref{fig:agents:comm-model}. A message exchanged between two agents 
in FIPA agent communication model consists of two parts (a) an envelop and (b) a content. 
%
The structure and the semantics of the envelop is defined by the FIPA Agent Communication Language (ACL). It contains a performative 
that signifies the purpose of the message. FIPA ACL defines 22 performatives, each of which belongs to a performative class described
in figure~\ref{fig:agents:performatives}. These performatives are used to implement different types of agent interactions. We shall 
explain some of them through examples in the following sections.
%
Besides a performative, the envelop of a message contains a few other fields, e.g. the identities of the sender and the receiver 
agents. A response message incorporates context, comprising a reference to the original query or request and the identity of the 
interaction protocol (discussed in the next section) to which it is a part of. 

\index{content language}
\index{RDF} 
The content field in a message can contain a simple yes/no type to response (to a confirmation request), or a more complex  
expression, like a plan to be executed or some knowledge to be shared. Different applications need different kinds of 
expressivity in the contents. Accordingly, FIPA standardizes a set of content languages for use in different contexts.
For example, {\em Semantic Language (SL)} can be used to formulate propositions like in predicate calculus, {\em Constraint
Choice Language (CCL)} is useful to model constraint satisfaction problems and {\em Resource Description Framework 
(RDF)}~\footnote{Described in chapter~\ref{chap:knowledge}.} can be used to communicate a knowledge graph. FIPA ACL
is open to new content languages to be incorporated in the standard. 
 
\index{ontology!application in agent communication}
Another important aspect of communication is establishing the formal meaning of the tokens used in the content. For example, there 
should be a common understanding between the sender and the receiver agents about meanings of the terms ``book'', ``price'' and
``Rs.'' (the unit of currency used), when the message content deals with the price of a book. The terms are generally domain-specific, 
and the sender and the receiver refers to an ontology (see chapter~\ref{chap:knowledge}) to establish a common meaning of the 
tokens. A FIPA message envelop (ACL) contains a pointer to the reference ontology for the content.

\subsection{Interaction Protocols}

\index{interaction protocols} 
In order to implement system functionality, an agent is expected to act and respond in a certain way on receipt of a message,
following some social norms. For example, on receiving a request, a receiver are generally expected to fulfill the request and
confirm, or let the sender know it's inability to do so.
%
While the interaction between two agents in a specific application is guided by the contextual needs, they can be composed
out of a finite set of generic building blocks. FIPA identifies a number of such elementary interaction units and standardizes
models for such interactions. We present an illustrative example for such interaction models.

\subsubsection*{Request Interaction Protocol}

\index{request interaction protocol}
{\it Request} is one of the simplest interaction protocols, where an agent $A$ sends a request to another agent $B$ and 
expects some response from $B$. A request can take two forms: (a) request for some information, and (b) request for some action to be 
undertaken taken. Agent $B$ may either refuse to act on the request, or agree to act upon it. If $B$ agrees to act, it's response
can be either (a) a piece of information that had been requested, or (b) a confirmation that the action has been taken, depending
on the request type. In either case, the agent may fail to satisfy the request. Thus, the possible message flows in the request
interaction can be as depicted in figure~\ref{fig:agents:request} as message sequence charts. 

\begin{figure}[!htpb]

	\subfigure[$B$ refuses to act on the request]{
		\begin{minipage}{0.45\linewidth}
			\centering
			\epsfig{figure="./Agents/request-refuse.eps",width=0.8\linewidth}
		\end{minipage}
	}
	\subfigure[$B$ fails to fulfill the request]{
		\begin{minipage}{0.45\linewidth}
			\centering
			\epsfig{figure="./Agents/request-failure.eps",width=0.8\linewidth}
		\end{minipage}
	}
	\\
	\vspace{2mm}
	\subfigure[$B$ confirms that the action has been taken]{
		\begin{minipage}{0.45\linewidth}
			\centering
			\epsfig{figure="./Agents/request-done.eps",width=0.8\linewidth}
		\end{minipage}
	}
	\subfigure[$B$ reports the results of a query]{
		\begin{minipage}{0.45\linewidth} 
			\centering
			\epsfig{figure="./Agents/request-result.eps",width=0.8\linewidth}
		\end{minipage}
	}

	\caption{Possible message flows in FIPA {\bf request interaction} protocol}
	\label{fig:agents:request}
\end{figure}


% \subsubsection{English Auction Interaction Protocol}
% 
% English auction is the auctioning process, where the seller sets an initial floor price of an item and announces it to
% a set of potential buyers, called the bidders. On receiving the announcement, the bidders quote a higher price that they
% are willing to pay for the commodity. The seller accepts the highest bid received, sets it as the new floor price, and
% repeats the cycle. When none of the buyers are willing to pay more price than the price offered in the earlier cycle,
% the auction ends, and the seller sells the item to the highest bidder. 
% The FIPA protocol (FIPA00031) for English auction is shown in figure~\ref{fig:agents:eng-auction}.
% 
% \begin{figure}[htbp!]
% 	\centering
% 	\epsfig{figure="./Agents/eng-auction.eps",width=0.4\linewidth}
% 	\caption{Message flow in FIPA {\bf English auction} protocol}
% 	\label{fig:agents:eng-auction}
% \end{figure}
% 
% An explanation of the message flow diagram is as follows. The seller agent multicasts the start of auction followed by the first 
% call for proposal (CFP) with a floor price $x$ to all potential buyers. The figure shows only one instance of buyer agents for 
% simplicity. The participating buyers asynchronously proposes higher values $x_1, x_2, \dots$ for the product. The seller accepts 
% the proposal of the buyer, who sends the maximum bid and rejects the others. Second round of auction starts with a CFP with a new 
% floor price $y=\text{max}(x_1, x_2, \dots)$. When there is no bid higher than the floor price received, the seller proceeds with
% making selling the commodity to the highest bidder with a REQUEST message (to pay the money), and informing others about the 
% closure of the auction. Note that the bid of a buyer made through a PROPOSE message represents a commitment of the buyer to buy the 
% product at that price, which it needs to honor. 
