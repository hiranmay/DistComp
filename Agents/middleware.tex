\section{Agent Middleware}

Agent middleware platforms are designed to support a set of functions pertaining to agent creation, destruction and migration,
collectively known as {\it agent management services}. Besides, they provide APIs for facilitating agent communication. There
are two distinct categories of agent middleware platforms: (a) those supporting homogeneous programming environment, such as
C/C++ or Java, and (b) those supporting multiple programming environments. Further, some of the platforms support 
agent migration. 

\subsection{FIPA Reference model}

\index{FIPA reference model}
In order to support multi-agent systems comprising agents hosted on different middleware platforms, interoperability of such
platforms need to be ensured. FIPA specifies some common management principles for the agent platforms to achieve such 
interoperability. FIPA reference model for an agent platform is shown in figure~\ref{fig:agents:ams}.

\begin{figure}[!htbp]
	\centering
	\epsfig{figure="./Agents/ams.eps", width=0.8\linewidth}
	\caption{FIPA Agent Management Reference Model}
	\label{fig:agents:ams}
\end{figure}

\noindent
As shown in the figure, a FIPA compliant agent platform minimally comprises the following components:
\index{agent management system} \index{directory facilitator} \index{yellow page} \index{agent communication channel}
\begin{itemize}
	\item {\em Agent Management Services (AMS)} to support creation and destruction of agents, and the migration of an agent 
		from/to some other node. Further, the AMS need to provide a unique name to an agent on creation, which can be 
		used to refer to it for communication purposes.
	\item {\em Directory Facilitator (DF)} (also known
		as a ``Yellow Page'' service) that maintains a list of all agents and their properties. It should provide search
		service to locate an agent (if available) with some specified capabilities.
	\item {\em Agent Communication Channel (ACC)} is a messaging platform that enables agents on the same platform (which can
		be distributed over multiple computing nodes) as well as on other platforms to communicate to each other over some 
		standard transport protocol, e.g. TCP or UDP.
\end{itemize}

\noindent
AMS and DF are implemented as FIPA compliant agents. Other agents can talk to them using ACL and relevant interaction protocols. 
For example, a newly created agent can register with the DF using ``Request'' interaction protocol. 

\index{FIPA agent life-cycle}
The life-cycle of an agent in FIPA reference model is shown in figure~\ref{fig:agents:life}. An agent, before it is created
in at an {\em unknown} state. It comes to an {\em initiated} state when created by the hosting platform, and to {\em active}
state when executed. It can move to {\em waiting} state to wait to be woken up by an external event. An agent can also be {\em suspended}
by the AMS. A mobile agent goes to the {\em transit} state while migrating. An agent in active state may quit, or it can be destroyed
by the AMS, taking it back to the unknown state.

\begin{figure}[!htbp]
	\centering
	\epsfig{figure="./Agents/life.eps", width=0.8\linewidth}
	\caption{Life-cycle of an agent}
	\label{fig:agents:life}
\end{figure}

\subsection{FIPA Compliant Middleware}

Several agent middleware have been developed during 1995  -- 2005, some of which are FIPA compliant. Two of the platforms,
which are FIPA compliant and are still in use, are JADE and MobileC. Both the platforms can be distributed over multiple 
computing nodes and support mobile agents. They can be used over a variety of platforms including constrained devices like
in mobile handsets and IoT based systems, besides personal computers and servers. We shall briefly introduce some salient 
design considerations of these platforms. Agent mobility will be discussed in the next section.

\subsubsection*{JADE: Java Agent Development Environment}

\index{JADE} \index{Java Agent Development Environment|see {JADE}}
JADE~\citep{Bellifemine:2005} is a FIPA compliant platform that has been developed to facilitate development of multi-agent
systems in Java programming environment. Use of Java programming environment naturally takes care of portability and security
issues in a multi-agent system. Since the FIPA ACL introduces significant overheads, JADE distinguishes between 
intra-platform and inter-platform messages. When two interacting agents are hosted on the same platform, JADE uses Java 
RMI for communication. When the two interacting agents are hosted on different platforms, use of FIPA-ACL becomes mandatory. 
Further, JADE incorporates a library of FIPA interaction protocols over the communication layer to enable quick application
development. 

Another important design consideration of Jade is to limit the number of threads running on a node. It uses {\em active object 
design}~\citep{Lavender:1996} principles, when each agent runs on a single thread. The tasks are classified based on their
behavioral pattern and are queued to run on asynchronous threads.
JADE-LEAP is an engineered version of Jade that can run on resource constrained devices, such as Java enabled mobile phones. For 
example, it has been ported to Android platform by wrapping it as a service~\citep{Bergenti:2014}. In this approach, a Jade container 
is split into a front-end running on the mobile terminal (where the JADE runtime is activated), and a back-end running on a remote
server (possibly hosted on the wired network). Multiple back-end modules can be created based on the workload of the agents.
% ~\footnote {The installation and programming manuals (with examples) for Jade are available on 
% \url{https://jade.tilab.com/documentation/tutorials-guides/}.} 


\subsubsection*{MobileC}

\index{MobileC}
Use of Java as the programming environment in Jade and many other agent platforms results in significant runtime overheads.
Mobile-C~\citep{Chen:2006} uses C/C++ as the programming environment to support of agent based systems on hardware with lower 
capability, e.g. the 
control systems, which are generally programmed in low-level languages. Use of a low-level language for agent programming
poses portability (due to difference in machine architecture) and security (e.g. recall power of pointers in C) issues. MobileC 
solves the problem by running the agents in an interpreted virtual environment. The virtual environment exercises complete
control over the agent and ensures portability, but arguably introduces some execution overheads as compared to the code run 
on a bare machine.

On MobileC platform, an agent is divided into multiple subtasks, which are listed in a task list. The task-list can be dynamically 
updated by introducing new subtasks and deleting the completed/aborted ones. A task from the list can be scheduled on different nodes 
of the system. The ability to dynamically change the task list and allocate the tasks on different nodes introduces flexibility 
in agent execution and performance improvement.

\subsection{Agent Migration}

\index{mobile agent!migration of} \index{strong migration} \index{weak migration}
Agent migration involves executing foreign code on a node and poses some unique challenges for the agent platforms, most
important ones being portability and security. A mobile agent, on migration to a new node, should have supporting runtime
environment. Further, a mobile agent need to be protected against a possible hostile environment and a host should be
protected against a hostile agent. As we have seen, both Jade and MobileC addresses the portability and security issues by 
running the agents in interpreted environment.

There are two types of migration supported by the different agent platforms. 
\begin{itemize}
	\item {\em Strong migration}, where the agent code, it's execution state and data are captured on the source
		node and migrated to a target node. After a strong migration, the agent is recreated on the target
		node and it's execution states and data are reinstated. The agent continues to execute it's code exactly
		from the same point and with the same data as  it had been interrupted on the source node. In general,
		it is not possible to support strong migration on a platform where the different nodes support heterogeneous 
		programming environment. For example, an agent coded in Java needs a JVM support to execute, and which
		may not be available on a target machine. 
	\item {\em Weak migration}, where only the agent data is transferred from one node to another. The agent code is
		assumed to pre-exist in the target node. The agent on the source node is terminated and a corresponding one
		is initiated when an agent ``migrates'' to the target node. Weak migration is possible across heterogeneous
		nodes, since different implementations of an agent in different programming environments, implementing the
		same logic, may be installed on the various nodes. Further, when the data volume is low, weak migration 
		imposes lower overheads than strong migration.
\end{itemize}

Of the platforms discussed, Jade supports strong migration and MobileC supports weak migration. We discuss the Jade agent
migration strategy here. Considering the nuances of heterogeneous platforms, FIPA interaction protocol for agent migration 
is under-specified. Jade has extended the FIPA interaction protocols to implement agent migration function~\citep{Ametller:2003}. 
In this model, an agent willing to migrate interacts with AMS using FIPA ACL, and the latter obliges by moving it.
%
The migration process consists of the following steps:
\begin{enumerate}
	\item The agent that wants to to migrate starts a conversation with the AMS agent of the source platform. It creates
		a description of itself (serialized code and data) and sends a ``Request'' message to the AMS with expressing
		it's intention to move, destination node and the description of itself.
	\item The local AMS on the source node may decide to accept or refuse the request. In the former case, it forwards the 
		request with the description of the agent to the AMS on the destination node. 
	\item The AMS on the destination node exercises it's own criterion to decide to accept or to reject the request. If either
		of the two AMSes rejects the request, the same is made known to the requesting agent and the process is aborted.
	\item If the AMS of the destination process accepts the request, the acceptance is made known to the requesting agent
		through the AMS on the source node. The AMS on the source node terminates the agent.
	\item The AMS of the destination node creates an instance of the agent and loads it's data (from the agent description)
		and executes it. The agent migration is complete.
\end{enumerate}

\noindent
A simplified interaction protocol for agent migration in Jade is shown in figure~\ref{fig:agents:migration}.

\begin{figure}[!htbp]
	\centering
	\epsfig{figure="./Agents/migration.eps", width=0.8\linewidth}
	\caption{Jade interaction protocol for agent migration}
	\label{fig:agents:migration}
\end{figure}

In Mobile-C environment, the mobility is introduced by scheduling the different subtasks of an agent on different nodes. The data 
of an agent is moved to the target node when the agent ``migrates''. Once a subtask has started on a certain host, the agent 
cannot move until the execution has been completed.

