\section{Applications of Agent Based Systems}

It is advantageous to model an application as an agent-based system, when the application involves dynamic coordination of
many system components, needs to process distributed and heterogeneous data, especially when the components or the data are distributed 
over a number of administrative domains. Agent based systems have evolved in task and mission complexity since mid-1990s. A few 
representative examples of agent based systems include business applications~\citep{Jennings:1996},
intelligent manufacturing~\citep{Shen:2006}, e-governance~\citep{Chaabane:2015},drone swarms~\citep{Shrit:2017,Lomonaco:2018}, 
coordination of multi-robot systems~\citep{Chen:2020}, and simulation~\citep{Davidsson:2007,Kim:2020}. In this section, we present 
a case study of application of agent based systems in multimedia information retrieval.

\index{information retrieval}
Multimedia artifacts pertaining to a domain, such as cultural heritage, are generally curated by many agencies and there and 
several independent archives. Being developed independently, the archives preserve different media forms, in different formats 
and with different schema for archival metadata. While the archives collectively present a rich information source, the 
fragmentation of information and heterogeneity of the resources present formidable challenge for semantic retrieval and/or 
browsing though manual or automated means.

An agent based architecture for integrating diverse multimedia archives of Indian Cultural Heritage has been presented 
in~\citep{Ghosh:2008,Chaudhury:2011}. Figure~\ref{fig:agents:heritage} depicts the interactions amongst the classes of 
agents in the system. In this architecture, each of the legacy repositories holding heritage data in
multimedia format has been encapsulated as an autonomous agent, called the {\em repository agents}. Some of the
repositories are based on multimedia databases, and implement media-based annotations (e.g. MPEG-7 descriptions) or indexing.
Each of the Repository Agents is knowledgeable about the metadata, indexing scheme and retrieval/browsing interfaces for the 
repository that it encapsulates. Further, the system is also populated with a set of mobile {\em media (processing) agents}, 
which can travel to the repository sites and perform specific media-based retrieval tasks (e.g. image search using a 
specific algorithm), thereby enhancing the retrieval capability of the repositories.

\begin{figure}[htpb!]
	\centering
	\epsfig{figure="Agents/heritage-architecture.eps", width=0.6\linewidth}
\caption{Agent-based architecture for Multimedia Information Retrieval for Indian Cultural Heritage}
\label{fig:agents:heritage}
\end{figure}

A user of the system presents his query to a {\em user agent}. The system is populated with a set of User Agents, each of which
implements one or a combination of different query paradigms, such as natural language text, query by example, etc. A user may
choose a specific query interface depending on his preferences or the application needs. A User Agent encodes a query in an 
internal query language that is expressive enough to support different query paradigms. 

In general, there may be a significant difference between the query constructs and the repository organizations. A user query 
needs to be reformulated for effective retrieval in a repository. A {\em coordinator agent} expands the query by collaborating 
with a set of {\em ontology agents}, which encapsulate knowledge about Indian cultural heritage, which is available in a fragmented 
manner in multiple administrative domains. The uniqueness of the ontology representation scheme used in this 
applications is it's capability to encode audio-visual properties of the concepts (heritage artifacts) and their spatial and temporal
structures over and above their semantic relations, and support for Bayesian reasoning model, which is more suitable for media
data interpretation~\citep{Chaudhury:2015}. 

The outcome of query expansion is a large and redundant set of elementary query components. In general, a repository agent
can service a subset of these specifications. The next step involves development of retrieval plans for each of the repositories, 
which is done through contract negotiation between the coordination and the individual repository agents. The repository agents,
in turn, can sub-contract some of the retrieval tasks to the available media processing agents. Thus, the contract negotiation
becomes a recursive process. Some of the eligible repositories are contracted the retrieval task by the coordinator after a 
cost-benefit analysis~\citep{Ghosh:2004}. Once the results are available from the repositories, the coordinator collates
them to produce the final retrieval results.

\index{semantic broker}
As explained in the paragraphs above, the system represents an example of distributed planning and data processing in multiple 
stages. The key capability of query reformulation in this architecture arises from query expansion and contract negotiation. 
The query expansion in this system is knowledge-based, as compared with combinatorial decomposition for federated SPARQL
query processing as discussed in chapter~\ref{chap:knowledge}. We have seen information broker architecture in 
chapter~\ref{chap:pubsub}. The coordinator performs the role of a {\em semantic broker} in this system.
Another major advantages of the agent based architecture is the dynamic team formation in individual query contexts. This 
enables the system to be dynamically populated or depopulated without service disruption, as well as dynamic load balancing.

