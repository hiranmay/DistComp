
\index{agent-based system} \index{multi-agent system|see {agent-based system}}
\index{autonomy}
As the complexity of the distributed systems grows, there is a need for making the computing elements independent in design and
autonomous in operations. Autonomy of the system components means that they should be able to execute without human intervention
for prolonged periods of time. It implies that, given some high level goal, the system components should be able to create their 
own plans, execute them and react to any unforeseen situations. This, in turn, requires that they be able to make independent 
decisions and that computing intelligence be distributed over the network. An intelligent and autonomous system component is 
called an {\em agent}. An agent-based system (also called multi-agent system) comprises multiple agents communicating with each 
other over a network. 
%
Today, the agent-based systems are used in many application scenarios, ranging from purely software environment such as information 
retrieval and e-commerce, to cyber-physical systems in industry, battlefields and scientific explorations. While the number of 
participating agents and decision-making capabilities (degree of autonomy) of the individual agents may vary from system to system, 
a common characteristics of the agent based systems is that each of the agents specializes in implementing a specific function. 
Agent based systems represent a bottom-up system design paradigm, with each agent designed independently to realize a specific 
capability. The system behavior emerges from interaction of a group of agents. 
For example, a unmanned ground vehicle may be designed to operate on a factory floor and have some generic capabilities like
autonomous navigation and carrying some objects from one place to another, without a the nature of the requirement of a specific 
factory unknown. In a warehouse environment, such vehicles can be assigned specific tasks like shelving incoming goods and
retrieving outgoing goods, to implement warehouse functionality. This gives a tremendous flexibility in deployment of these agents 
over traditional distributed systems, where the interaction of system components are pre-defined and tailored to the system goals. 

In this chapter, we begin with a brief introduction to intelligent agents and their internal architectures. We introduce multi-agent
systems and continue with the realization of their social abilities in a distributed environment. At the outset, we present the 
protocols for communication and interaction of the agents. We discuss a reference model for agent platform and agent mobility.
Following this, we present methods for coordination of autonomous agents, planning and contract negotiation that are important
issues for realizing system functionality in an agent based systems. We discuss the security issues posed by agents and platforms
of heterogeneous origins and sketch the methods to address them. We provide an example application of agent-based system as a use-case
and conclude the chapter with some salient observations.

\input ./Agents/architecture.tex
\input ./Agents/communication.tex
\input ./Agents/middleware.tex
\input ./Agents/coordination.tex
\input ./Agents/applications.tex
\input ./Agents/conclusion.tex
\input ./Agents/exercises.tex

