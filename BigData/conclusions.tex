\section{Conclusion}

\index{big data system}
In this chapter, we have presented several techniques to store, access and process large volumes of data in a distributed environment. 
The three Vs of {\em big data}, namely volume, veracity and velocity pose formidable challenges to processing such data. Besides,
processing of unstructured and dynamic data calls for techniques, different from those deployed in traditional data processing 
systems.

We have presented architectures for storage systems, namely RAID, SAN and Cloud, which comprise many independent storage units
and can scale up to peta-bytes of data. Data redundancy in these systems provide fault-tolerance. We have reviewed the 
architectures of two different distributed file systems, namely Ceph and HDFS, that provide UNIX-like interface and transparent 
access over distributed storage. These file-systems ensure high system availability. No-SQL databases are built over these 
file systems to store and manipulate unstructured data. We have exemplified parallel processing over these databases with
map-reduce and Pregel algorithms.
%
Further on, we have presented a generic architecture, the lambda architecture, which is commonly applied in distributed analytics 
systems. The architecture enables integration of static data stored in databases and dynamic data that are streamed. We also
discuss distributed algorithms for two popular clustering techniques, namely k-means and DBScan, and the BIRCH algorithm for
stream clustering.

In this chapter, we have focused on the distributed computing technology that forms the foundation for distributed data analytics,
which is indeed a vast subject and there are many books that provide more insights into the techniques for distributed analytics.
Discussion of those techniques are beyond the scope of this book. Further, this chapter deals primarily with data at their lowest 
level, and does not try to organize them to abstracted knowledge. We shall deal with such abstraction process and distributed knowledge 
management in the next chapter.


