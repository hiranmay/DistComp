
\section*{Exercises}

\begin{enumerate}
	\item Hand simulate and verify the super-steps in Pregel algorithm as shown in figures~\ref{fig:bigdata:Pregel-supersteps}(a)--(f).
		Is it possible for a vertex to go inactive in one super-step, but to become active again at a subsequent super-step?
	\item Consider the following data-points in a two-dimensional space: 
		$(1.9,0.9)$, $(0.9,0.8)$, $(2.2,1.3)$, $(1.8,1.2)$, $(1.9,1.8)$, $(2.1,1.7)$, $(1.8,2.2)$, $(1.1,0.8)$, $(2.1,0.9)$, 
		$(0.8,1.2)$, $(0.9,1.7)$, $(2.2,2.2)$, $(1.1,1.9)$, $(1.2,2.3)$, $(1.2,1.2)$, and $(0.8,2.1)$. 

		\begin{enumerate}
			\item Write a program for K-Means algorithm and partition the data into two categories. Report the number
				of iterations it takes to complete.
			\item Choose different seeds for the initial cluster centers randomly, and explain the results.
		\end{enumerate}
	\item Implement the peer-to-peer architecture for distributed K-Means algorithm~\citep{Bandopadhyay:2006} 
		(section~\ref{sec:bigdata:d-kmeans}) over Pregel algorithm (algorithm~\ref{algo:bigdata:pregel}). Define the
		processing, and the input and output messages for the super-steps.
	\item Update DBScan algorithm (algorithm~\ref{algo:bigdata:dbscan}) to report ``core points'' and ``border points''
		of a cluster.
	\item For MR-DBScan algorithm (section~\ref{sec:bigdata:mrdbscan}), what should be the optimal depth of halo and inner halo 
		for a subspace?
	\item In BIRCH algorithm (section~\ref{sec:bigdata:stream}), when two clusters are merged, what will be the parameters for 
		the CF in the merged cluster?
\end{enumerate}
