Trust in the large distributed systems, participated by many agents in an open environment, is a serious concern. 
Traditionally, trust is delegated to the agent
% ~\footnote{In this chapter, we use the term ``agent'' to mean 
% an individual, an organization, or an automated hardware/software entity that works on behalf of an individual or an organization.
% In general, the agents are self-interested, i.e. they behave rationally to maximize their own benefits.} 
that has the administrative 
responsibility for operating the system. In many applications, trust is generated by maintaining an authoritative {\em ledger} that
records the sequence of all transactions by all participating agents in the system. Any dispute in the system is resolved by 
referencing the ledger. For example, the bank is the the trusted party in a banking system and the ledger maintained by it
is the authority for resolving any dispute. The trusted party needs to validate and authorize the transactions before including
in the ledger. In order to enable the validation, the participating agents need to entrust sensitive personal data, such as identity
and banking passwords, with the trusted party. 
This approach suffers from over-reliance on a single system component, which may behave unethically or be vulnerable to security 
attacks.  A breach of trust, either accidental or by a deliberate malicious action, can be harmful to the stake-holders of the 
system. Moreover, a central hub for validation of transactions may prove to be a bottleneck or point of failure in many distributed 
transaction systems.
%
\index{distributed trust} \index{distributed ledger}
This motivates solutions to build ``trust'' democratically over multiple agents in a distributed peer-to-peer system. The core 
idea is to create a {\em distributed ledger}, whose maintenance responsibility is shared democratically by the participating agents. 
In this chapter, we shall deal with the technology to build distributed ledgers, to deal with the security threats in multi-agent 
systems introduced in the last chapter. 

We begin the chapter with a brief overview of the cryptographic techniques that are the fundamental building blocks for building 
distributed ledgers. This is followed by the essential properties of the distributed ledger systems and a generic architecture. 
In the next section, we discuss blockchain technology and the Bitcoin, which is by far the most popular of the distributed ledger 
technologies and cryptocurrency applications respectively, as on date. Next, we introduce some alternative consensus protocols 
and datastructures, such as tangle and hashgraph, that overcome limitations of blockchain, and are gaining ground in some application 
domains. Further, we move on to scripts and smart contracts that enable distributed control, and illustrate with execution of a 
distributed plan. In the upcoming 5G era, cyber-physical systems are going to represent very large distributed systems touching 
many aspects of human life and society, where privacy and security will be of paramount importance~\citep{Fan:2018}. We review the 
adaptations of blockchain technology to support such applications in the next section. 
Finally, we conclude the chapter with an evaluation of contemporary technologies for distributed trust against the various application 
requirements.

\input ./Ledger/crypto.tex
\input ./Ledger/dledger.tex
\input ./Ledger/blockchain.tex
\input ./Ledger/dconsensus.tex
\input ./Ledger/smart-contract.tex
\input ./Ledger/cyber-physical.tex
\input ./Ledger/conclusion.tex
\input ./Ledger/exercises.tex
