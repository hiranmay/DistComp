\section{Distributed Ledger Systems}

\index{ledger}
Several applications need to keep track of sequences of transactions. Common examples of such systems are banking system that 
keeps track of monetary transactions by it's customers, and supply chain management systems that track movement of assets across
locations and change of ownership. The document that chronicles the transcactions in a system is called a {\em ledger}. It should
be possible for the records in a ledger to reliably recalled at a later date. Therefore, the records should not get deleted or 
or modified, either accidentally or by a malicious action of an agent.

\index{ledger|textbf}
\begin{definition} [ledger]
	A {\em ledger} is a data structure consisting of an ordered list of transactions. A transaction is a completed agreement 
	between two agents to exchange goods, services, or financial assets. A ledger should be persistent and secure.
\end{definition}

\index{distributed ledger|textbf} \index{gossip protocol}

If a ledger is to be maintained without a central trusted party, a natural solution demands that each 
(or, many) of the agents may maintain a copy of the ledger. In open systems, where the agents can arbitrarily join or withdraw from 
the network, a gossip-like protocol (see chapter~\ref{chap:gossip}) may be used to percolate transaction information to all nodes in 
the system, so that each can update it's copy of the ledger. The foremost benefit of maintaining a distributed ledger is that the 
system becomes persistent. An accidental or malicious corruption of data at one node can be recovered by use of voting in a 
democratic setup. Since every agent has access to a copy of the ledger and can verify the validity of the records, there is a greater 
transparency in the system. If one, or a few, colluding agents produce a dishonest version of the ledger, it will be voted out.
Further, the computational bottleneck of a centralized book-keeper is avoided. Even if some of the nodes in the system become
overloaded (or fails), the overall system performance will not degrade. 

\index{distributed ledger|textbf}
\begin{definition} [distributed ledger]
	A {\em distributed ledger} is an authoritative sequenced and permanent set of records collectively held by a significant 
	number of participating agents at any point in time. It is a replicated data-structure which can only be appended to.
\end{definition}

\index{distributed consensus|textbf}
\index{BFT}
A key challenge in maintaining a distributed ledger is that of {\em distributed consensus}, i.e. how all the nodes agree on a common 
and honest version of the ledger, despite some of the nodes being faulty or dishonest. The consistency of the ledger across the nodes 
need to be maintained not only in terms of the contained records, but also their sequence, which the network delays can influence (see
chapter~\ref{chap:event-ordering}). In general, the problem is known as Byzantine fault-tolerance (BFT)~\citep{Lamport:1982} and has 
been discussed in chapter~\ref{chap:agreement}. 

\begin{definition} [distributed consensus]
	Given that there are $n$ agents, each with an input and assuming that a minority of the agents may be faulty or dishonest, 
	{\em distributed consensus} refers to a protocol, that results in (a) termination with all honest nodes in agreement for 
	the value of the input, and (b) the agreed upon value is generated by one of the honest nodes.
\end{definition}

\index{distributed ledger system|textbf}
\begin{definition} [distributed ledger system]
	A {\em Distributed Ledger system} is a system of electronic records that enables independent agents to establish a consensus 
	around a shared ledger, without relying on a central coordinator to provide the authoritative version of the 
	records.~\citep{Rauchs:2018}.
\end{definition}

Many algorithms have been proposed to address consensus in a distributed system, which is the key concern of a distributed ledger. A 
commonality of all the methods is that they rely on cryptographic techniques for various purposes. The key differences amongst the methods 
are computational complexity, latency, data-structure, openness and security level, which make each of the technologies suitable for a
different application domains.  

\subsection{Properties of Distributed Ledger Systems}

In essence, a distributed ledger system is a distributed database system (see chapter~\ref{chap:bigdata}) with the difference that there 
is no central control. The properties of the system can be summarized as follows:

\begin{enumerate}
	\item {\em Shared record-keeping}: It should enable several participating agents to collectively create, maintain,
		and update a shared ledger that comprises a sequenced set of authoritative records.
	\item {\em Multi-party consensus}: It should enable all participating agents to come to an agreement on a shared
		set of records (the ledger). 
	\item {\em Independent validation}: It should enable each participating agent to independently verify the state of 
		the transactions and integrity of the system.
	\item {\em Persistence, tamper-evident and tamper-resistant}: The ledger should be replicated over multiple nodes for 
		persistence. It should be extremely hard for an agent to tamper with the ledger (tamper resistant). In the
		unlikely event of the ledger being tampered with, it should be possible to detect any tampering easily (tamper-evident).
	\item The system can be either {\em permissionless} or {\em permissioned}.
		\begin{itemize}
			\item A permissionless system is an open system, where any agent can freely participate without authorization 
				from any central agency. The participating agents can by fully anonymous. Generally, they have a high 
				churn, i.e. many agents joining or leaving the network over a period of time, and their structure is 
				dynamic. Such systems generally operate in public domain or applications spanning across multiple 
				administrative domains. 
			\item In a permissioned system, only the agents identified and authorized by a central authority can join. 
				Such systems are relatively stable with less churn, and are generally private, i.e. confined to an 
				administrative domain.
				%
				Further, the permission can either be {\em read-only}, when the agent can observe the ledger without 
				modifying it, or {\em read-write}, when the agent can both read and modify the ledger. Fine grained
				access control is also possible.
		\end{itemize}
\end{enumerate}



\subsection{A Framework for Distributed Ledger Systems}

While the diffetrent distributed ledger technologies may use different datastructures and algorithms, \citep{Rauchs:2018} identifies 
three generic interdependent layers for a distributed ledger system as shown in figure~\ref{fig:ledger:dledger}.  A brief description 
of the layers are as follows:

\begin{figure}[!htbp]
	\centerline{
		\epsfig{figure="./Ledger/dledger.eps", width=0.8\linewidth}
	}
	\caption{A framework for a distributed ledger system}
	\label{fig:ledger:dledger}
\end{figure}

\begin{enumerate}
	\item{\em The protocol layer}. It defines a set of protocols (software-defined rules) that determine how the system operates. 
		We can identify two major components in this layer. 
		\begin{itemize}
			\item {\em Genesis component} comprises a set of initial code-base that defines the protocols. It also contains 
				a genesis record that forms the seed of the ledger. 
			\item {\em Alteration component} deals with evolution of the protocol over time. It includes the governance 
				aspects (i.e.  processes to arrive at collective decisions regarding changes) as well as an implementation 
				consideration (i.e.  processes to implement the changes over the network).
		\end{itemize}


	\item{\em The network layer}. It interconnects the participating agents and the processes that implement the protocol. 
		They are three major components in this layer.  
		\begin{itemize}
			\item {\em Communications Component}: It specifies which agents can participate in the network, access privileges 
				for the data and authorization for initiating transactions.
			\item {\em Transaction Processing Component}: It comprises a set of processes that specifies the mechanism for
				updating the shared ledger. It includes the policies for (a) which of the agents have the right to 
				update the ledger, and (b) how participants can reach agreement over implementing these updates.
			\item {\em Validation Component}: It deals with the process for verification of the compliance of the transactions 
				and the data with the protocol.  
		\end{itemize}

	\item{\em The data layer}. It deals with the data flowing through the system, and their semantics in specific contexts with 
		respect to the system. There are two distinct components in this layer.

		\begin{itemize}
			\item {\em Operations Component}: The processes which determine the data that should be used for creating new 
				records, modifying existing records, and the methods for doing so. 
			\item {\em Journal Component}: This is metadata for the ledger, e.g. which records are in a block, what is the 
				sequence of the records, etc.
		\end{itemize}

\end{enumerate}

