\section{Cryptographic Techniques}

This section provides a concise review of the core cryptographic techniques that are essential to build distributed 
ledger systems.  For more details of cryptographic methods, a reader is recommended to consult any standard book on cryptography.

% \subsection{Hashing}

\index{hashing|textbf}
\begin{definition} [hashing]
	{\em Hashing} is defined as an one-way mapping from a data-space $x$ to a hash-space $y$, $y = H(x)$. The reverse mapping
	$x = H^{-1}(y)$ does not exist.
\end{definition}

\index{hash function}
% A simple hash function for a set of data, each 1024 bit long, could be to retain its last sixteen bits, i.e. $y = H(x) = x \% (2^{256})$, 
% where `\%' stands for modulo operation. Note that it is possible to compute $y$ given any $x$, but not vice-versa. However, a hash function 
% in this form is not particularly useful for cryptography. 
A cryptographic hash function needs to satisfy a few properties, namely
collision resistance, data hiding and puzzle friendliness. 

\index{collision resistance|textbf}
\begin{definition} [collision resistance]
	A hash function $H()$ is said to be {\em collision resistant} if it is infeasible to find two values, 
	$x$ and $y$, such that $x \neq y$ and $H(x)= H(y)$.
\end{definition}

If the hash-space is smaller than the data-space, collision is bound to happen. A hash function need to be so designed that it is 
not possible to make an intelligent guess about colliding data values. In other words, given a data value, $x_1$, a colliding data 
$x_2 \neq x_1$ can be found by trial and error only; If the hash space is of size $2^{256}$, it will take $2^{128}$ attempts.
% ~\footnote
% {A quantum computer needs $O(\sqrt(N))$ operations to find a colliding data value, when $N$ denotes the size of the data 
% space~\citep{Brassard:1997}, prompting research on quantum computation resistant hashing algorithms~\citep{Fernandez-Carames:2020}.}. 
To realize the enormity of the number, assume that we can compute $2^{20}$ ($\approx$ one million) hashes per second, when on the 
average, finding an instance of colliding data will take $2^{108}$ seconds, which is more than $10^{25}$ years! 

\index{data hiding|textbf}
\begin{definition} [data hiding]
	A hash function $H()$ is said to be {\em data hiding}, if when a secret value $r$ is selected from a high entropy 
	probability distribution function, then given $y = H(r \mid \mid x)$, it is infeasible to find $x$. 
	(A high entropy probability distribution function is one where there the probability value over any specific range is not 
	significantly higher than others. This means that there is little knowledge about the distribution. Uniform probability 
	density function has the highest entropy.)
\end{definition}

\noindent
The use of the string $r$ in the above definition calls for an explanation. If the data-space is small, it is not possible to 
achieve data hiding with any hash function $H(x)$. Hash can be computed for each possible data value, and the one that 
matches $y$ is a possible solution. The way to achieve data hiding is to artificially expand the data space by concatenating
the data with a secret string $r$, and then computing $y = H(r \mid \mid x)$. The secret string is chosen from a high entropy 
probability distribution function, so that it's value cannot be guessed.
%
\index{nonce}
While $x$ cannot be recovered from $y = H(r \mid \mid x)$, it is possible to authenticate a claimed copy of $x$ against
the original, by computing it's hash and comparing it with the known value of $y$. Note that the string $r$ needs to be 
disclosed during the authentication process. The string $r$ is referred to as the {\em nonce}, since it can be used only 
once. 

\index{puzzle friendly|textbf}
\begin{definition} [puzzle friendly]
	A hash function $H()$ is said to be {\em puzzle friendly}, if for any possible $n$-bit output value $y$, if $r$ is chosen 
	from a high-entropy probability distribution function, then it is infeasible to to find $x$ such that $H(r \mid \mid x) = y$ 
	in time significantly less than $2^n$.
\end{definition}

The computational puzzle deployed in this context involves finding a suitable nonce $r$, such that $H(r \mid \mid x) = y$.
It can be done only with trial-and-error method. In practice, $y$ is constrained to a narrow range of the hash-space. For example, 
with $y$ restricted to $1/10^{20}$-th of the hash space, an agent, on the average, will have to make 50 quintillion (trillion 
trillion) attempts to find a suitable value for $r$.

\index{SHA-256}
A commonly used hash function that satisfies these properties and meets the security needs of most of the applications is
SHA-256.
The function takes arbitrary long input strings and produces a 256-bit output. Fast implementations of the algorithm are 
available on FPGA~\citep{Suhaili:2017} and ASIC~\citep{Zhang:2019}, which are used in distributed real-time systems. 
% \hgcomment{need a good citation for SHA-256}

\index{digital signature|textbf}
\index{PKE}
% In a centralized trust system, a central agency is entrusted with some private information (e.g. a password) about the 
% participants, which is used to authenticate the identity of the initiator of a transaction. 
In absence of a central trusted party in a peer-to-peer system, every agent should be able to authenticate a transaction without 
accessing any private information of the initiator. A solution to the problem is in the form of a {\em digital signature}, which 
is based on public key encryption. 

\begin{definition} [digital signature]
	A {\em digital signature} is a unique bit-pattern appended to a message, which is an irrefutable proof of the
	signatories endorsement of the message. 
\end{definition}

In digital signature, an agent $i$ signs a message created by it using it's secret key $sk_i$. Moreover, the signature string is a 
function of the message, over and above the secret key, so that it cannot be copied from one message to another. Other agents
can verify the authenticity the signature using the public key $pk_i$ of agent $i$.  
%
\index{ECDSA}
\index{elliptic curve digital signature algorithm|see {ECDSA}}
The sizes of the keys and the signature strings determine the level of security of a specific implementation. An implementation 
that is widely used is the {\em Elliptic Curve Digital Signature Algorithm (ECDSA)}~\citep{Johnson:2001}, which uses a private
key of 256 byes and produces a signature string of 512 bytes.


