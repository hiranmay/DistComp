\section*{Exercises}

\begin{enumerate}
	\item Is it possible to keep the nonce value alone (and not the computed hash value) in the hashtag of a 
		blockchain?
	\item What will be the implications, if individual records (rather than blocks) were chained like in a Blockchain
		to create a distributed ledger?
	\item Analyze the distributed consensus process described in section~\ref{sec:ledger:nakamoto-consensus}. Relate
		the various aspects of the process with the three layers and components of the DLT system framework.
	\item Assume that a new transaction (10) is included in the tangle shown in figure~\ref{fig:ledger:tangle} and
		that it approves nodes 6 and 7. What will be the new tips of the network? Assuming that the weight of each
		node is 1, what will be the cumulative weight for the node 3 in the network after node 10 is added?
	\item Review the code for smart contract for sealed bid first price auctions available at 
		\url{https://vyper.readthedocs.io/en/stable/vyper-by-example.html}. 
		Adapt the code for creating a smart contact for sealed bid seconf price (Vickrey) auction~\footnote{A description
		of Vickrey auction is available at 
		\url{https://saylordotorg.github.io/text\_introduction-to-economic-analysis/s21-04-vickrey-auction.html}.}.
		Try it out on Ethereum (see \url{https://ethereum.org/en/developers/learning-tools/}).
	\item Consider the distributed ledger shown in figure~\ref{fig:ledger:tangle} in form of a tangle. Assume that a new 
		transaction (11) is addded to the ledger, and it approves the transactions 6 and 9. What will be the new tips in 
		the ledger? What will be the cumulative weight of node 2, assuming the individual weights of each of the nodes to 
		be 1?
\end{enumerate}
