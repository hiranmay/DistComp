\section{Conclusion}

As distributed systems grow in their scale and cover every aspect of life, from financial transactions to control of household
gadgets, there is a growing fear among public about misuse of private data and of malicious attack on such systems, which is 
not unfounded. Distributed trust mechanism, where the consensus is reached in a democratic manner, is likely to play an important 
role in designing such systems. The enormous success of Bitcoin cryptocurrency and blockchain technology, is an evidence of public's 
confidence on such financial systems. The application of distributed ledger technology, in particular blockchain have been extended 
beyond cryptocurrencyty applications, in domains such as e-governance~\citep{Batubara:2018}, industry~\citep{Bodkhe:2020}, 
healthcare~\citep{Mettler:2016} and various cyber-physical systems, e.g. smart homes~\citep{Dorri:2017}, smart city~\citep{Panarello:2018}, 
smart grid~\cite{Li:2018}  and vehicle-to-vehicle communication~\citep{Elagin:2020}.
%
The prime perceived advantage of distributed ledgers is that they are not in control of any single administrative domain and cannot 
be compromised with a single point of malicious attack, and even limited collusion. Use of cryptographic techniques in these ledgers 
ensures irrefutable proofs of transactions, yet preserving privacy of the participating agents. 

The currently available technologies for establishing distributed trust do not satisfy all application demands and are still at
an evolving stage. Notably, anonymity and immutability of records, which are the main strengths of distributed ledgers do not 
satisfy government regulations with respect to some applications. Anonymous financial transactions can fund drugs, terrorism 
and such other activities harmful to the society, and is frowned upon by most of the governments. As another example, DLT cannot 
prevent an incorrect record (possibly generated by a honest agent through mistake) to be included in a ledger, and once included,
it will be stored in the ledger for ever. Laws in some countries require such records to be permanently erased on detection; 
correction through a supplementary record is not sufficient in some applications.

A closer look at the technologies may point out that these technologies are not really as ``democratic'' as they are promised
to be. The code that implements the datastructures and consensus protocols are always in control of a handful of system designers
and programmers, though the open source codes can be to publicly audited. Further, there can always be a concentration of power in
a consensus protocol. For example, about 75\% processing power of Bitcoin is concentrated with five mining pools~\citep{Ferdous:2021}.
This means that though everybody is authorized to mine blocks, very few people can do it in practice. Also, a possibility of collusion
and corrupting the Bitcoin network in favor of a few cannot be ruled out. Other forms of consensus, such as PoET, runs the risk
of trusting some specific hardware. As the distributed ledger technology is embracing many applications other than cryptocurrencies,
the major research question is that if it is possible to have a simpler model for distributed database based on decentralized trust,
and this is probably the time to rethink the principles behind distributed ledger technology~\citep{Kuhn:2019}.
