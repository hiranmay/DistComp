\section{Other Techniques for Distributed Consensus}

We have discussed the Nakamoto distributed consensus algorithm based on PoW in the previous section. The protocol is used
in a very large majority of applications~\citep{Ferdous:2021}, including Bitcoin. Nevertheless, the technology has it's own 
limitations and there has been several attempts to improve on it or substitute it for specific applications. In this section, 
we review the alternative data-structures and algorithms for distributed consensus.


\subsection{BFT and PoW}
\index{BFT} \index{PoW}

% \hgcomment{Ratan, please review the section carefully for correctness.}

We have discussed {\em Byzantine Fault Tolerance} (BFT) in chapter~\ref{chap:agreement}. Like PoW, BFT also achieves integrity 
of data in distributed systems, despite faulty or malicious nodes. The major differences in the two approaches are as 
follows~\citep{Vukolic:2016}.

\index{gossip}
BFT relies on synchronous communication between the agents for seeking votes. Thus, it needs to assume a permissioned system 
comprising agents with known identities and little churn. On the contrary, PoW based consensus works for open systems, where 
the information is asynchronously percolated in the network through gossip, and it is sufficient for an agent to communicate
to a few neighbors (at least one) in the network.
%
Further, the synchronous message communication in BFT results in an overhead of $O(n^2)$ ($O(n^3)$ where protocols require 
acknowledgement for votes), which restricts it's scalability (tested with $\leq$ 20 nodes). In
contrast, asynchronous operations in blockchain provide excellent scalability (in order of thousands of nodes). 
%
Further, while BFT can tolerate up to 1/3rd of the total nodes to be corrupt, collusion of more than half of the available computing 
power is required to defeat the PoW-based distributed consensus mechanism. 
 
Where BFT scores over PoW is in it's throughput, latency and energy efficiency. PoW consumes lots of computing power (and electricity)
and hence is extremely slow and wasteful. In comparison, BFT can achieve excellent throughput (in order of several thousand transactions 
per second) and achieves latency comparable to that of the network.
 
Thus, we find that the Nakamoto protocol is more suitable for an open permissionless network environment with slow transaction rates. 
BFT protocol is more suitable for use in small permissioned networks requiring a higher throughput. Some variations of Byzantine fault 
tolerance (BFT) algorithms, e.g.~\citep{Castro:2002}, are used in small and private networks. We review some alternative ``proofs'' 
that attempts to replace the expensive PoW algorithm in the next section.

\subsection{Alternative Proofs}

\index{PoW!memory-bound algorithms}
\index{incentive-based proofs}
The PoW algorithm proposed by Nakamoto is compute-bound, a barrier than can be overcome with deployment of advanced processor
technologies. For instance, quantum computer needs $O(\sqrt(N))$ operations to find a colliding data value, when $N$ denotes 
the size of the data space~\citep{Brassard:1997}. This has motivated research on quantum computation resistant hashing 
algorithms~\citep{Fernandez-Carames:2020}. One approach to overcome the problem is to use memory-bound PoW algorithms, e.g. 
Dagger-Hashimoto algorithm~\citep{Buterin:2013}, where the memory access rate limits the performance. Though it can be argued 
that the improvement in memory technology can break this barrier too, it may take a long time considering the current technological '
trends. 
%
\index{PoS|textbf} 
\index{proof of stake|see {PoS}}
As an alternative to PoW, Etehereum uses {\em proof-of-stake} (PoS) algorithm,
% ~\footnote{see Proof-of-Stake (POS): \url{https://ethereum.org/en/developers/docs/consensus-mechanisms/pos/}.}, 
where the honesty of an agent is ensured with the stake it has in contributing a block. The PoS algorithm is better energy efficient
than PoW, since it does not require an agent to solve a puzzle, but rather to stake a part of his assets for block mining. The average 
time to generate a block in Ethereum is about 10 second, in contrast with 10 minutes in Bitcoin, which uses PoW. Further, since it does 
not require heavy investment in computing resources there is a lower barriers to entry, and reduced concentration of mining power. 
 
\index{BFT}
\index{PoET|textbf} \index{proof of elapsed time|see {PoET}}
\index{incentive-less proofs}
The PoW and PoS algorithms are incentive-based and are suitable for cryptocurrency applications. In applications divorced
from cryptocurrencies, there is no financial motivation for the agents to invest in producing a proof. Quite a few incentive-less 
consensus algorithms have been developed for such applications. For example, Hyperledger Sawtooth and some other protocols use 
a proof of elapsed time (PoET),
% ~\footnote{See PoET 1.0 specifications:
% \url{https://sawtooth.hyperledger.org/docs/core/releases/latest/architecture/poet.html}.}, 
where an agent needs to wait for a minimum
time to create a block. The algorithm is based on a trusted hardware environment (Intel SGX) and can be suitable for permissioned 
distributed ledger systems. The duration of the elapsed time can be adjusted to balance between the application throughput needs and 
the consistency of the ledger.

\section{Non-linear Datastructures}

In applications, where there are no incentive for block mining, a pragmatic approach for such applications is to dispense with the 
blocks and allow the agents to append the transactions directly to the ledger. To avoid the performance bottleneck caused by the 
linear stucture of a ledger (as in blockchain), which becomes more severe while operating at the transactions level), these 
protocols use non-linear datastructures where the ledgers can grow in parallel with periodic synchronization. We review two distinct 
protocols based on this approach in this section. These protocols achieve much higher throughput than blockchain, in order of 100,000 
transactions per second.

\subsection{Tangle}
\index{tangle}
\index{DAG} \index{directed acyclic graph|see {DAG}}
Tangle~\citep{Popov:2018} is a data-structure designed for distributed ledger of IOTA, cryptocurrency to support micro-payment in IoT 
based systems. The transactions in tangle are organized as a connected {\em Directed Acyclic Graph} (DAG),
% ~\footnote{A directed acyclic 
% graph (DAG) is a directed graph, where there are no cycles, i.e. no directed paths from a node to itself. A connected DAG means that 
% it is possible to visit any node in the graph from any other node, ignoring the directivity of the edges.}, 
a copy of which is maintained
by every participating agent in the system.
%
A tangle is initialized with a genesis transaction, where all coins (or assets) are generated and assigned to the different agents
in the system. 
For all transactions, the initiator agent is required to {\em approve} at least two existing transactions in order to place a 
transaction in the graph, with the obvious exception that the very first transaction has only the genesis record to approve. 
While approving a transaction, an agent checks if the transaction is consistent with the history in the tangle, and creates 
appropriate hashes. The work done by the agents in approving the transactions contribute to the network's security, and determines 
the {\em weight} of a transaction.  An agent is motivated to do an honest ``work'', because it runs the risk that other agents 
will not approve it's transaction otherwise.

Figure~\ref{fig:ledger:tangle} shows the structure of a tangle with a few transactions. In the figure, a node represents 
a transaction and an edge represents an approval. The transaction numbers are shown in their chronological order. Each transaction 
(except 0 and 1) approves exactly two existing transactions. If there is a direct link from a transaction $u$ to a transaction $v$, 
$v$ is {\em directly approved} by $u$. If there are a set of transactions $\{ z_1, z_2, \dots, z_n \}$ and links 
$z_n \leftarrow u, z_{n-1} \leftarrow z_n, \dots, v \leftarrow z_1$, $v$ is {\em indirectly approved} by $u$. 
%
In the figure, transaction 7 directly approves transactions 3 and 5, and indirectly approves transactions 0, 1 and 2. It does not 
approve (either directly or indirectly) transactions 4 and 6. Transaction 0, the {\em genesis node}, is approved by all other 
transactions, either directly or indirectly. 
%
At any given point of time, there are a few transactions in the network that are yet to be approved, for example transactions 6, 
9 and 10 in the figure. An unapproved transaction is called a {\em tip} of the graph.

\index{tangle|textbf}
\begin{definition} [tangle]
	A {\em tangle} is a connected directed acyclic finite graph $G = \langle V,E \rangle$ where each vertex $v \in V$ represents a 
	transaction and $(v \leftarrow u) \in E$ represents an approval. The following properties hold good for the graph: 
	\begin{enumerate}
		\item $\exists v_0 \in V$, where $v_0$ is the genesis transaction.
		\item $deg_{out} (v_0) = 0$ 
		\item $\forall v \in V \setminus \{ v_0 \}, deg_{out} (v) \geq 1$ 
		\item $\forall u,v \in V$ if $\exists (v \leftarrow u)  \in E$, then $u$ represents a transaction prior to $v$  and 
			that $u$ has ``approved'' $v$, and
		\item $\forall v \in V \setminus \{ v_0 \}$, either  $v_0 \leftarrow v$, or 
			$\exists \{z_1, \dots, z_n\}$ such that $z_n \leftarrow v, z_{n-1} \leftarrow z_n, \dots, v_0 \leftarrow z_1$ 
	\end{enumerate}
\end{definition}

The {\em status} of the tangle at time $t$ is given by $G(t) = \langle V(t), E(t) \rangle$, where the following properties hold good.
\begin{enumerate}
	\item At time $t = 0$, $G(0) = \langle \{v_0\}, \emptyset \rangle$.
	\item $\forall t_1,t_2 \geq 0$, if $t_1 > t_2$, then $V(t_1 ) \subseteq V(t_2)$ and $E(t_1) \subseteq E(t_2)$.
\end{enumerate}

% A {\em tip} of a tangle is a node $v \in V$, for which $deg_{in}(v) = 0$.

\begin{figure} [!htbp]
	\centerline{
		\epsfig{figure=Ledger/tangle.eps, width=0.8\linewidth}
	}
	\caption{The structure of a tangle}
	\label{fig:ledger:tangle}
\end{figure}
 
\index{cumulative weight}
A transaction that is approved in tangle is trusted by the participants. The degree of trust increases with the number of direct
and indirect approvals. Formally, the degree of trust is determined by the {\em cumulative weight} of the transactions that is 
defined as 

\begin{equation}
	cw(v) = w(v) + \sum_{u \in \text{approver}(v)} w(u)
\end{equation}
\noindent
where $w(x)$ denotes the weight of a transaction $x$ and  approver$(v)$ represents the set of all transactions that approves 
transaction $v$ directly or indirectly. For example, assuming the weight for each of the transactions to be 1, the cumulative 
weight of transaction 5 is 5.

Thus, to build trust in the transactions, it is necessary that (a) each transaction is approved by many later transactions, and (b) 
there are as few unapproved nodes at any given point of time as possible. Each transaction is created as a tip (not approved by any 
other transaction) and remains so until the time certain existing transactions approve it. Ideally, an agent creating a new transaction 
should approve the current tips of the DAG. A random walk algorithm in the DAG from a random node towards the tips ensures that an
agent selects a tip in the tangle. 
%
Because of network delays, an agent may not see the current tips, but the tips that have been there at a previous point of time.
It can be proved that the network remains stable when this strategy is followed by all the agents, i.e. every transaction gets 
eventually approved and the number of tips in the DAG fluctuates around a value given by
\begin{equation}
	\hat{L}^{(k)} = \frac{k \lambda h}{k - 1}
\end{equation}
\noindent
where each transaction approves $k$ transactions, the transactions are assumed to be generated as a stochastic process with an arrival 
rate of $\lambda$, and the network delay is $h$, i.e. a transaction attached to the tangle at time $t$ becomes visible to the 
network at time $t + h$. 
%
In a {\em low load regime}, when the transaction rate and network latencies are small ($\lambda h$ is small), few tips can be expected
in the network at any given point of time. On the contrary, in a {\em high load regime} ($\lambda h$ is large). there will be many tips.
 
\index{double spend attack}
The consensus algorithm is also effective against forks in the tangle, which can be caused by statistical behavior of the system or 
dishonest behavior by an agent (double-spend attack). A fork results in a split in the tangle, with two branches becoming disjoint 
after a certain point of time. The random walk algorithm prefers following a ``heavier'' branch (having transactions with more 
cumulative weights) in the network, which is likely to include honest transactions, when there is a double-spend attack. It can be 
shown that the optimal strategy for self-interested (not dishonest) agents is to follow the protocol, thereby ensuring cooperation 
with the network~\cite{Popov:2019}. 

\hgcomment{What are the significance of height, depth and score? Do we discuss them?}

\subsection{Hashgraph}

\index{hashgraph} \index{gossip}
Hashgraph consensus algorithm~\citep{Baird:2016:1} is based on history of gossip protocol. At initiation, each of the participating 
agents creates an event. As the time progresses, a member chooses another member at random, and they exchange information about all of 
the events that they know. The information exchange is optimized by first finding out what the other agent already knows, 
and then providing the unknown information only. The agents communicate with gossip protocol; whenever an agent becomes aware of 
a new event, it spreads the information through the community until every agent becomes aware of it. 

\begin{figure}[!htbp]
	\subfigure [] {
		\begin{minipage}{0.45\linewidth}
			\centerline{
				\epsfig{figure=Ledger/hashgraph.eps, width=0.9\linewidth}
			}
		\end{minipage}
	}
	\subfigure [] {
		\begin{minipage}{0.45\linewidth}
			\centerline{
				\epsfig{figure=Ledger/hashgraph-event.eps, width=0.8\linewidth}
			}
		\end{minipage}
	}
	\caption{(a) Structure of a hashgraph, and (b) structure of an event in a hashgraph}
	\label{fig:ledger:hashgraph}
\end{figure}

\index{gossip!history of}
The history of gossip can be depicted as a graph as shown in figure~\ref{fig:ledger:hashgraph}(a). In the figure, the system
is participated by four agents, designated by $A$, $B$, $C$ and $D$ respectively. The vertical lines represent the timeline for each, 
with time increasing upwards. The nodes in the graph marked $a_0, \dots, d_0$ represent the initial events.  Other nodes, 
$a_1, b_1, \dots$ represent events acknowledging an information exchange. For example, $d_2$ represent a mutual information exchange 
($B \xrightarrow{b_1} D, D \xrightarrow{d_1} B$) between agents $B$ and $D$, initiated by agent $B$, and acknowledged by agent $D$.
%
While in gossip protocol, only the information exchanged and not the history of gossip is preserved, the ledger in hashgraph 
consists of the graph itself for traceability of the information exchange. 

The structure of an event in a hashgraph is shown in figure~\ref{fig:ledger:hashgraph}(b). It contains a time-stamp for the 
information exchange and the hash of the two reference events, e.g. event $d_2$ contains the hash of events $b_1$ and $d_1$. 
Further, an agent can optionally include one or more new transactions signed by itself (payload) that it wants to be included 
in the ledger. Use of cryptographic hash and signature makes the older parts of a hashgraph tamper-resistant. 

If two agents have the same version of the hashgraph, then they can arrive at identical order of the events, which is a 
deterministic function of that version of the hashgraph. However, at any given point of time, the older events will be known 
to all the participating agents, though the most recent events may not yet be known. For example, in the current state of 
the hashgraph shown in figure~\ref{fig:ledger:hashgraph}(a), the events $b_1$ and $d_1$ are known to all the agents, 
though agents $B$ and $C$ is not aware of the event $a_1$. Thus, the ordering of the events will be identical for
the part of the hashgraph that has propagated to all agents. The ordering of the recent events will be limited to the events
known to an agent, which will be different for the different agents, but not in contradiction with each other.

\index{hashgraph!consistency}
\begin{definition} [consistency of hashgraphs]
	Hashgraphs of agents $A$ and $B$ are consistent, if for any event $x$, if both hashgraphs contain $x$, then they both 
	contain the same set of ancestors for $x$, with the same subgraph for those ancestors.
\end{definition}

As in other distributed ledgers, there can be a fork in a hashgraph because of a double-spend attack. A dishonest agent creates
two events $x$ and $y$, none of which is an ancestor to the other. Each of the events has a common ancestor, say $z$, and attempts 
to transact the same coin (asset). In such cases, some of agents could accept the event $x$ and some could accept $y$.
%
For achieving Byzantine fault tolerance, there need to be a consensus in the network to accept one of the events and not the other. 
The consensus in hashgraph is achieved with a {\em virtual voting} algorithm. Since each participating agent stores a copy of the 
hashgraph, an election conducted by an agent does not require communicating to other agents seeking votes as in BFT algorithm, but 
references to it's own hashgraph only. Before we proceed to describe the virtual election algorithms, we define a few concepts of 
hashgraph.

\begin{definition} [see]
	Normally, an event $x$ can {\em see} an event $y$ if $y$ is an ancestor of event $x$. If some event $w$ has has both $x$ 
	and $y$, but neither of $x$ and $y$ is an ancestor of the other (double-spend attack), then by definition, $w$ does not see 
	either of the two events. 
\end{definition}

\begin{definition} [strongly see]
	If there are $n$ participating agents, an event $x$ can {\em strongly see} an event $y$, if $x$ can see at least $\frac{2}{3}n$ 
	events by different agents, each of which can see $y$. 
\end{definition}

In figure~\ref{fig:ledger:hashgraph}(a), event $d_3$ can see $b_1$. Since $n=4$ in the example, an event $x$ needs to see at least 
$3$ events, each of which sees $y$ in order that $x$ can strongly see $y$. In the example, $d_4$ sees $a_1$, $b_2$ and $d_3$, each of 
which sees $b_0$. Thus, $d_4$ strongly sees $b_0$. The reader is encouraged to verify that $d_4$ can strongly see $c_0$ and $d_0$
also, but not $a_0$.  
 
An agent, after creating an event, executes three algorithms. The first algorithm defines the {\em rounds} in a hashgraph and the
{\em witnesses} in every round. 

\begin{definition} [round number]
	The {\em round number} of the initial events in the hashgraph is defined as 1. It is incremented by 1, when an event that can 
	strongly see events created by at least $\frac{2}{3}n$ agents in the previous round. 
\end{definition}

\index{witness!in hashgraph}
\begin{definition} [witness]
	The first event (for each agent) in a round is called a {\em witness}. The initial events, which do not have any parents (first
	events of round 1), are also called the witnesses. 
\end{definition}

\begin{figure}[!htbp]
	\centerline{
		\epsfig{figure=Ledger/tr2-hashgraph.eps, width=0.4\linewidth}
	}
	\caption{An extended hashgraph (\reproducedp{~\citep{Baird:2016:2}})}
	\label{fig:ledger:tr2-hashgraph}
\end{figure}

\noindent
We extend the hashgraph (and change the labels) as in figure~\ref{fig:ledger:tr2-hashgraph} to illustrate these concepts. 
In the figure, the rounds are annotated. The labelled events are are the witnesses in each round. Each of these events can 
strongly see events created by at least 3 agents in the previous rounds.
 
The second algorithm decides the fame of the witnesses through virtual voting. The votes are cast by the witness events only.

\index{famous witness!in hashgraph}
\begin{definition}[famous witness]
	A witness is {\em famous} if it is seen from witnesses of at least $\frac{2}{3}n$ agents in the next round.
\end{definition}

\noindent
For example, in the figure, all of the witnesses A3, B3, C3 and D3 can see B2. Hence, B2 is a famous witness. Event C2 is not seen by 
A3, B3 and D3; it is not famous. The reader is encouraged to verify that the witnesses A2 and D2 are also famous.

Once the famous witnesses are discovered, the third algorithm defines the order of the events. An event $x$ receives a round
$r$ if that is the lowest numbered round in which at least half the famous witnesses could see it. For example, the event just above
A1 in figure~\ref{fig:ledger:tr2-hashgraph} is seen by 2 famous events (A2 and D2) out of 3, and hence receives round 2. If an 
event $x$ receives round $r_1$ and an event receives round $r_2$, then $x$ precedes $y$ if $r_1 < r_2$, and vice-versa. If two events 
$x$ and $y$ receives the same round, they are ordered based on the median of the timestamps of the events where each famous witnesses’s 
creator first had an event that saw $x$ and $y$ respectively.

With this consensus algorithm, it can be proved that (a) hashgraphs maintained by all the agents will be consistent, and (b) if event 
$x$ is a fork with event $y$ and $x$ is strongly seen by event $w$ in hashgraph of agent $A$, then $y$ will not be strongly seen by any 
event in any hashgraph of agent $B$, if the two hashgraphs are consistent. This ensures Byzantine fault tolerance. Note that the voting 
algorithms are based on $\frac{2}{3}$rd majority and hence it takes more than $\frac{1}{3}$rd agents to be dishonest to corrupt the system.

% \subsection{Summary}

% Taxonomy~\citep{Garay:2020}
