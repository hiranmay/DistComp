\section{Distributed Ledgers for Cyber-physical systems}

Cyber-physical systems are large distributed systems, with several thousands of nodes, distributed over large geography and
operating over multiple administrative domains. Many of the applications carry sensitive personal or process data, and building 
privacy and data security in such systems is of paramount importance. Distributed ledger technology offers an effective trust 
mechanism in such systems. The main applications of distributed ledgers in cyber-physical systems are in providing (a) a secure 
storage for all transactions, (c) access control to the devicesand (b) smart contracts enabling secure transaction flow. In this 
context, we use the word ``transaction'' with a generalized meaning of any action taken by an agent in the system, and not only 
exchange of money or other assets. 
%
% There has been several proposals for using blockchain, which is by far the most popular of distributed ledger technologies, in diverse 
% cyber-physical systems, e.g. smart homes~\citep{Dorri:2017}, e-healthcare~\citep{Liang:2017}, smart city~\citep{Panarello:2018}, smart
% grid~\cite{Li:2018}  and vehicle-to-vehicle communication~\citep{Elagin:2020}. 
% \hgcomment{Check on Tangle and Hypergraph}
 
Application of blockchain technology to cyber-physical systems brings in some unique opportunities as well as some unique 
challenges~\citep{Reyna:2018}. The major advantages include (a) decentralization and scalability, (b) identity with anonymity, and 
(c) reliability and security, which we have discussed in the earlier sections. 
%
\index{PoET} \index{hashgraph} \index{tangle} \index{permissioned systems}
A major challenge in application of blockchain technology in IoT systems is posed by the compute intensive PoW-based consensus algorithm,
which IoT systems and ill-affort and for which there is no motivation. The IoT-based systems are generally permissioned systems, and 
limited degree of trust in certain system components are often acceptable. At the same time they demand higher transaction rates to
be supported. They generaly deploy alternative proofs (PoET, in particular) and non-linear data-structures such as tangle and hashgraphs 
are often used~\citep{Khor:2021}. 

\subsection{Layered architecture}

Another issue pertaining to deployment of distributed ledger technology in cyber-physical systems are that most of the IoT 
devices are constrained in terms of memory and processing power, and cannot take up any significant computing activity. 
Further, the transactions in a cyber-physical system mostly consist of exchanging device data, and that the IoT devices in 
many applications generate very large volumes of data. A distributed ledger requires all transaction data to be permanently 
stored. The monotonic increase in the stored data requires expensive storage and can become unmanageable. 

\index{blockchain!layered architecture}
To overcome these limitations, blockchain technology has been adapted to suit the cyber-physical applications. The basic approach 
is to organize the system in two layers, where the ledger is maintained by the higher level nodes with more computing power. Further, 
while the history of the transactions are preserved (who communicated to whom and for what purpose) to achieve traceability, the 
actual data exchanged are discarded. These adaptations exploit the fact that the 
IoT devices can be grouped into distinct administrative domains in most of the applications, and each domain is controlled by one
or more cloud-connected servers. For example, each home constitutes an administrative domain, where the connected devices are
controlled by a server in the domain in a smart home application (see figure~\ref{fig:ledger:smart-home}). The devices in an 
administrative domain are geographically collocated and are connected to each other over local wired or wireless network. The external
connectivity from the administrative domain are usually in control of the central server. Moreover, many of the transactions 
are local within an administrative domain, while some are across the domains. 

\begin{figure}[!htbp]
	\centerline{
		\epsfig{figure="Ledger/smart-home".eps, width=0.8\linewidth}
	}
\caption{Overlay in a typical smart-home network}
\label{fig:ledger:smart-home}
\end{figure}

The security in such environment can be organized in two layers, the local clusters and an overlay layer. In the local cluster,
where all devices are in the same administrative domain, the trust is delegated to a local manager (LM). Since the IoT devices have
limited processing power, encryption of local transactions are done with symmetric keys, i.e. a shared key between the requester
and the requestee, generated by a trusted party, namely the local manager. Every device in the local cluster registers with the
LM, when it's privileges are defined and it receives a unique key to interact securely with the LM. The local manager saves the 
local transactions in a secure centrally maintained ledger, relieving the devices from the task. 
%
The overlay layer accounts for secure transactions across the clusters. All incoming and outgoing transactions to and from
a cluster are controlled by an agent. These agents collectively maintain a lightweight blockchain for the inter-cluster transactions,
and are called the block managers (BM). The local manager (LM) and the block manager (BM) may be hosted on the same machine,
but are functionally different. Like in a standard blockchain, the transactions in the overlay layer are signed by public keys. 

\index{lightweight blockchain} \index{blockchain!lightweight}
The {\em lightweight blockchain architecture}~\citep{Dorri:2019} supports primarily two types of  transactions: (a) {\em store}, where 
a requesting device asks some data (generated by it) to be stored, and (b) {\em access}, where a device requests the data stored 
by another device. The device data can be stored either in the local storage of a cluster, or on cloud. If the data is to be stored 
locally, the requester agent sends a signed request to the LM, which checks the privileges of the requester, and if satisfied, creates 
a symmetric key and shares it with the requester agent and the requestee (storage) agent, and creates a local transaction record. 
The requester interacts with the requestee using the shared key. In contrast to a conventional blockchain record, the requestee agent 
also signs the transaction when the request is fulfilled. Similarly, for a local data exchange, i.e. for access to some data generated 
by a local device, the LM mediates between the requester and the requestee and the transaction is signed by both the parties. All 
transactions originating from a requester are chained with a genesis record that is generated when a device registers with the 
local manager.

An inter-block request is satisfied in three stages. First, the requester agent sends the request to the LM in it's own cluster, 
which connects the requester to the BM of the cluster (as the requestee), which forms a local transaction. Next, the BM, on 
receiving the request, creates an inter-block transaction record and broadcasts it on the overlay network. Finally, the BM of the 
target cluster (acting as a requester), connects to the target device (requestee) with the help of the LM of the cluster, which 
forms another local transaction. 

\index{consensus period} \index{utilization ratio}
\index{PoET}
To maintain the integrity of the overlay transactions, The BMs aggregate the transactions in blocks and collaboratively maintain a 
blockchain. The system uses a time-based proof (PoET) instead of PoW for consensus. To protect the overlay against a malicious BM, 
each BM is allowed to generate only one block during a {\em consensus-period}, which is at least twice the maximum end-to-end delay 
in the overlay network. It allows each block manager to validate a block and place it in the blockchain before the next block arrives. 
The consensus-period is dynamically adjusted to keep the {\em utilization ratio}, the ratio of the total number of new overlay 
transactions generated to the total number of transactions added to the blockchain, within a range. In a conventional blockchain, 
a block is verified by verifying all transactions contained in it. To improve real-time efficiency, the ``trust'' in a block manager 
is computed as a function of valid and invalid blocks it has generated in the past, and a fraction of the transactions are validated 
based on the trust value. 

In the overall architecture, the data flow is kept separate from the transactions (requests and their fulfillments). For example,
in a ``store'' transaction, the data to be stored is sent directly to the requestee, separated from the transaction (request to store)
that is routed via the LM. Similarly, an ``access'' transaction results in the requestee device sending the data to the requester 
directly, after the request is validated by the LM.
The separation allows the transaction records to remain compact, and obsolete data to be deleted though the transactions can still
be traced. Further, data can be accessed in real time rather than suffering block time delay. 

\subsection{Smart Contract in IoT Systems}

\index{smart contract}
The IoT systems often deal with sensitive personal information (e.g. those generated by wearable devices) and their access control
is of paramount importance. A smart contract that enables access control of the devices over a two-layer network architecture similar
to the one discussed in the previous section has been proposed in~\citep{Novo:2018}. 
%
A contract operates in the overlay layer and are executed by the block managers (BM). A device can be under control of one or more 
BMs. A manager or a device is identified with a dynamic list of public keys. The dynamic association between a manager and a device 
is represented by a dynamic binding between any of their public keys. This implies that a device may be known to different managers, 
or to the same manager at different times, with different keys, and vice-versa, which enhances the anonymity of the devices as well 
as the managers. A binding between a manager and a device (identified with public keys) is associated with an access control list, 
which is used by the manager in context of a transaction. This provides a flexibility that the access privileges of a device may 
vary from manager to manager and time to time, depending on the public key used in a given context.
%
The smart contract comprises the following functions:
\begin{enumerate}
	\item {\em Set-up}: A smart contract is created by an agent. A single smart contract defines all access control operations
		in the system. The address of the smart contract is broadcast to all the managers.
	\item {\em Registration}: The managers register themselves or other managers in the contract. They also register IoT devices 
		to be under control of one or more managers. 
	\item {\em Management modifications}: All the registrations described in the previous function can be updated.
	\item {\em Policy definitions}: A manager can define or modify the access control policies for all devices registered with them. 
		Since a device can be registered with more than one manager, and each manager can define it's own access rules, 
		it is possible to have an agent to have different access privileges to a device, by accessing it through different 
		managers. 
	\item {\em Transactions}: A device, when needs to access data from another device, forwards the request to a manager (need 
		not be the one controlling the requester device). The manager executes an operation in the contract to fetch the 
		requested data, if permitted by the access control policies, and returns it to the requesting device. The interaction 
		between the device and the manager is out of purview of the contract.
\end{enumerate}

Note that a device never requests an information from the blockchain in this approach. The separation of the data-flow and the 
block-chain processing results in device data to be available to another device in real time, and the corresponding transaction
record is eventually preserved in the blockchain.
