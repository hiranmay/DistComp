\section{Scripts and Smart Contracts}

\index{scripts}
So far, we have viewed a transaction in a distributed ledger as a record of coins (or assets) changing hands. In practice, the 
transactions are encoded as scripts that can validate the inputs and change the ownership of the coins. As an example, the output 
of a transaction in Bitcoin, rather than simply containing the public key of the new owner of the coin, specifies a script that 
asserts that the coins can be further redeemed by matching the signature with the public key of the new owner. Similarly, the 
input of a transaction specifies a script that provides the public key of the existing owner. The two scripts are merged and 
executed to validate the transaction.


\begin{figure}[!htbp]
	\centerline{
		\epsfig{figure="Ledger/state-machine.eps", width=0.8\linewidth}
	}
	\caption{Distributed ledge as a state transition machine}
	\label{fig:ledger:state-machine} 
\end{figure}

\index{state machine}
Thus, we can consider a distributed ledger as a state machine, where the states comprise the set of all unspent coins and their 
respective owners at a certain point of time. A transaction represents a state transition function. It takes an initial state 
($S_0$) and a transformation rule ($TX$) specified through a script. It validates $TX$ with respect to $S_0$, and (if valid) applies 
$TX$ on $S_0$  to produces a final state $S_1$. If validation fails, it produces an error and the state of the ledger does 
not change.  

\begin{equation}
	Apply(S_0, TX) \rightarrow S_1 (\text{ or, } ERROR) 	
\end{equation}

\index{smart contract}
The model is illustrated with a simple Bitcoin transaction in figure~\ref{fig:ledger:state-machine}, where an agent $A$
pays 10 BTC to agent $B$. A transaction can be viewed as a {\em contract}, which when executed changes the state of the ledger. 
The scripting language supported with Bitcoin is restrictive and supports transactions between two public key owners only. 
Later implementations of distributed ledgers provide more powerful scripting languages that can be used to create various 
applications based on distributed consensus. A contract created with a script is called a {\em smart contract}, since they 
can execute automatically, on being invoked under certain conditions, without needing human intervention. The word 
``smart'' in smart contract do not have any artificial intelligence connotation. 

% \subsection{Ethereum and Scripting}

A turing-complete scripting platform had been first  introduced with Ethereum, 
%~\footnote{see Ethereum white-paper \url{https://ethereum.org/en/whitepaper/}.}, 
where the sate of a ledger is represented by a set of accounts. An account is characterized by, amongst other things, an owner 
and the current account balance.
% ~\footnote{Ether is the native currency of Ethereum.}. 
An Ethereum account can send messages to other accounts with transaction requests (signed data packets).
%
\index{accounts in Ethereum}
Ethereum maintains two types of accounts: 
\begin{enumerate}
	\item An {\em externally own account} is a conventional accounts held by an agent. It is  controlled by the agent's
		private key. An agent holding an externally owned account can send signed messages. 
	\item A {\em contract account} contains some executable code (for the contract) and an internal storage that maintains 
		it's state. When a contract account receives a message, it's code is executed. It can, in turn, send messages 
		to other contracts as a result of the execution of its code. A contract account can be created by anyone.
\end{enumerate}

\index{Vyper} \index{Solidity} \index{Ethereum virtual machine}
There are several high-level scripting languages for coding a contract for Ethereum, of which Vyper (similar to Python) and Solidity
(similar to Java) are most popular. 
% {Example codes in Vyper are available at
% \url{https://vyper.readthedocs.io/en/stable/vyper-by-example.html}, and in Solidity at
% \url{https://raw.githubusercontent.com/ethereum/solidity/develop/docs/examples/blind-auction.rst}.}.
A script in any of the languages are compiled to a low-level language to be run on the Ethereum
virtual machine. 

%
% \index{gas in Etehereum}
% The scripting languages of Ethereum (and many other distributed ledgers) allow loop. This may result in very large (or, infinite)
% long execution times because of inadvertent programming errors or deliberate denial-of-service 
% attack. Ethereum solves this problem by making an account pay (with ether) for the steps executed. The account need to define
% a maximum amount (gas), it is willing to pay for the execution, it the gas gets exhausted before completing a transaction,
% the execution terminates and the state of the system does not change.  

% We take a simplified version of sealed-bid auction, described in section~\ref{sec:ledger:auction}, as an example 
% to smart contract and illustrate it with a pseudocode in listing~\ref{lst:ledger:auction}.
% 
% \begin{lstlisting} [basicstyle=\footnotesize, numbers=left, stepnumber=1,caption=Pseudocode for first-price sealed bid auction,
% 	label=lst:ledger:auction]
% # Data-structures
% auctioneer:address  # Auctioneer 
% endTime:time        # Auction end time 
% ended:boolean       # Auction ended  
% bids[]:array of structure # Received bids 
%    bidder:address   # Bidder 
%    value:number     # Bid value
% nbids:number        # number of bids received so far
% highestBid: structure # Highset bid received
%    bidder:address   # Highest Bidder (address)
%    value:number     # Highest bid value (number) 
% 
% # Initialize an auction
% init(_auctioneer:address, _bidTime:interval)
%    auctioneer = _auctioneer
%    endTime = now + _bidTime
%    nbids = 0
%    ended = false
% 
% # Bid with a value
% public bid(_bidder:address, _value:number)
%    if(ended) 
%       return error    # bid not accepted
%    else
%       bids[nbids].bidder = _bidder  # record current bider and bid-value 
%       bids[nbids].value = _value    # a bidder can submit multiple bids
%       nbids = nbids + 1             # increment number of bids received
%       return success   # bid accepted
% 
% # End auction
% # Return highest bidder and bid to auctioneer
% public endAuction() 
%    if (ended) or (now <= endTime) 
%       return error   # auction already ended, or it is not yet time
%    else
%       ended = true   # no more bid submissions allowed
%       highestBid = bids[argmax(Bids[].value)] # Bidder and value of the highest bid received
%       send(auctioneer, highestBid)
%       return success # auction successfully ended
% \end{lstlisting}
% 
% A brief explanation of the code is as follows. Lines 1--11 declares the various data structures that will be required
% for conducting the auction. The text following ``:'' indicates data type, e.g. auctioneer is of datatype address (public key).
% The text following ``\#'' is comment. 
% %
% The first function to be called is ``init()'' (lines 13--18), which creates and initializes the auction. It notes down the auctioneer, 
% computes the end-time for the auction, initializes number of bids received to zero, and declares ``ended'' to be false (assuming that 
% the end time is some time in the future). The function is executed only once (like a constructor in java). 
% %
% The function ``bid()'' (lines 20--28) accepts bid-values. It can be called by anybody by sending a message to the contract account. 
% The function records the bid value against the bidder in an array and increments the count.
% %
% The function ``endAuction()'' (lines 30--38) ends the auction and returns the highest bidder and the bid to the auctioneer. 
% Before doing so, it checks if the auction has already ended or the ending time has been reached.
% %
% The code is simple~\footnote{In the presented pseudocode for sealed bid auction, the bids of the early bidders can be reviewed
% by the late bidders, and the auction can be manipulated. A key feature in sealed bid auction is that the bidders encrypt their
% bids and reveals it when the auction duration has ended (see discussions in section~\ref{sec:ledger:auction}). We have not
% shown the ``reveal()'' function for the sake of brevity in this example. Complete code for sealed bid auction is available at
% \url{https://vyper.readthedocs.io/en/stable/vyper-by-example.html} (in Vyper), and at
% \url{https://raw.githubusercontent.com/ethereum/solidity/develop/docs/examples/blind-auction.rst} (in Solidity).}.
% and is trivial to implement on a centrally trusted system. The difference in implementing it on Ethereum is
% that every transaction will be accepted by global consensus, recorded in a blockchain and will be available for publicly audit
% at any point of time, thereby alleviating the possibility of manipulation.
% 
% \subsection{Distributed Plan as Smart Contract}

% Cyber-physical systems generally comprise large number (in order of thousands) of devices, often distributed over large 
% geographical area and are sparsely connected through unreliable networks. Nevertheless, the applications of such systems
% are often life-critical. A centralized control of the devices are impractical because of numbers, geographical distribution, 
% privacy requirements and unreliability of networks. It is customary in a cyber-physical system to design autonomous agents 
% to control a group of possibly collocated devices. The agents need to coordinate with each other to achieve some system
% goals~\citep{Ding:2019}. We have introduced distributed planning and execution in chapter~\ref{chap:agents}. In this 
% section, we introduce a smart contract based mechanism for establishing trust in such large-scale cyber-physical systems. 

\index{distributed plan}
\index{smart contract!distributed plan}
We illustrate the use of smart contract with execution of a distributed plan in agent based systems, for which have provided a 
representation in chapter~\ref{chap:agents} (see figure~\ref{fig:agents:mas-plan}). A successful execution of the plan requires 
that the actions be scheduled only when the input dependencies and the preconditions are met. 
For example, $act_{22}$ depicted in the figure can be executed only when $act_{11}$ has been completed and the preconditions 
$p(act_{22})$ are met. In a distributed system, it is possible that an agent executes an action prematurely, either by error 
or for some selfish reasons. 

\begin{lstlisting} [caption=Pseudocode for distributed plan execution, label=lst:ledger:plan]
init(plan)
   ...
   act_22:
      in:  [ act_11 ]  
      out: [ act_13, act_23 ]
      p:   [ p_x:(d_x,min_x,max_x), ... ] 
      e:   [ e_y:(d_y,min_y,max_y), ... ]  
      code: some code
   ...

execute( act ):
   # Check if input dependencies met 
   if !(act.in: is empty)
      return error
   # Check if all preconditions are satisfied 
   for each p_i in  act.p: 
      if !(check( p_i:(d_i,min_i,max_i) ) == success)
         return error
   # Dependencies and preconditions met -- run action code 
   run(act.code)
   # Check effects
   for each e_i in  act.e:
      if !(check( e_i:(d_i,min_i,max_i) ) == success) 
         return error 
   # Update preconditions for actions in output list
   for each act_i in act.out:
      remove act from act_i.in: 
   # All done
   return success

check( d_i, min_i, max_i ):
   s_i = read(d_i)
   if (s_i < min_i) or (s_i > max_i)
      return false;
   return true;
\end{lstlisting}

Correct execution of a distributed plan can be ensured with a distributed consensus mechanism~\citep{Shukla:2018}, without relying 
on a trusted scheduler, which can be the target of a security attack. In this approach, the distributed plan may be represented 
as a smart contract. A partial pseudocode for the contract is shown in listing~\ref{lst:ledger:plan}. In the listing, 
we have represented $act_{11}$ as \texttt{act\_11}, etc.  
%
At initialization time, the plan can be expressed as a list of actions, where each member of the list contains lists of 
dependencies (``in'' and ``out'' lists), preconditions and effects. It also contains the identity of the executable code  
for the action. For example, lines 3--8 in the listing depicts the entry for action $act_{22}$.
%
Execution of an action in the plan is represented by a function ``execute()'' (lines 11--29), which checks the input dependencies and 
the preconditions before running the corresponding code. Once the code is run, the (expected) effects are checked. Finally, current 
action (which has now been completed) is removed from the input list of all actions in the output list of the current action to update 
the depencies of the latter.  Thus, an action can be said to have it's input dependencies satisfied when it's input list is empty.

The preconditions and the effects are typically checked by verifying if some sensor readings are within certain range. Thus,
a precondition $p_i$ (or, an effect $e_i$) which is based on the reading $s_i$ of device $d_i$ can be expressed as

\begin{equation}
	p_i = s_i  \text{ in range } [ min_i, max_i]
\end{equation}

\noindent
In general, the device to read for determining the precondition or the effect of an action can be under the control of a
different agent. Thus, the checking of preconditions and effects can be represented as a function in the contract as shown
in lines 31--35. 

Besides reducing security vulnerability, another advantage of smart contract is that an agent is identified in the system through 
public keys only and the privacy of the agent is preserved. This can be of paramount importance in many systems, for example, in 
healthcare systems, dealing with private information of a person collected from wearable devices.

