\section{Distributed Knowledge Representation}
\label{sec:knowledge:dk}

\index{RTM}
\index{representational theory of mind|see {RTM}}
The term ``knowledge'' has been defined in many different ways in literature, some of which are philosophical and some are
technical. Use of knowledge in computing demands that the knowledge be formally encoded and be processed with formal algorithms.
For our purpose, we define knowledge is an organization of data that is an abstracted representation of a domain. We consider
knowledge to be composed of named concepts, and that the names can be used to create propositions. A set of propositions create
a domain model, from which conclusions can be drawn. This is in compliance with the Aristotelian view of {\em Representational 
Theory of the Mind} (RTM)~\citep{Stanford:2020}.

\index{distributed knowledge}
{\em Distributed knowledge} in a group of agents is the aggregate of the knowledge possessed by the individuals agents in the group. 
Intuitively, the agents in a community can have access to the distributed knowledge by pooling their individual knowledge through 
some communication mechanism. Formally, distributed knowledge can be defined as:

\index{distributed knowledge|textbf}
\begin{definition} [distributed knowledge]
	Let $\mathcal{G}$ be a group of distributed agents with cardinality $n$. Let, $\phi_1, \phi_2, \dots, \phi_n$ represent the
	knowledge possessed by each of the agent in group $\mathcal{G}$. The {\em distributed knowledge} of the group $\mathcal{G}$ is
	defined as $\phi$, iff $\bigcup_{1=1}^n \phi_i \rightarrow \phi$.~\citep{Roelofsen:2007}.
\end{definition}

\index{common knowledge}
In this context, another term that is used for the knowledge of a group is {\em common knowledge}, which is defined as the knowledge 
possessed by every member of the group, with an additional requirement that every member knows that others also know it. While 
distributed knowledge of a group is weaker (broader) than individual knowledge common knowledge is stronger (narrower) that individual 
knowledge. We shall see application of common knowledge in chapter~\ref{chap:agents}.  

\subsection{Resource Description Framework (RDF)}

Resource Description Framework (RDF)~\citep{Decker:2000} is a datamodel standardized by World-Wide Web Consortium (W3C) for 
distributed knowledge representation on the web. The basic unit of knowledge representation in RDF is a {\em triplet}, comprising 
a {\em subject}, a {\em predicate} and an {\em object}.

\index{RDF triplet|textbf} 
\begin{definition} [RDF triplet]
  		$RDF triplet := \langle subject, predicate, object \rangle$. 
\end{definition}

\noindent
In an RDF triplet, the subject and the object in the statement represent the two resources that being related, and the 
predicate (attribute) represents the nature of their relationship. The relationship represented by the predicate is 
directional, from the subject to the object. In other words, an RDF statement asserts a property of the subject, with
the predicate defining the attribute and the object defining the property value.

\index{RDF statement|textbf}
\begin{definition} [RDF Statement]
	Asserting an RDF triplet makes it an {\em RDF statement}.
\end{definition}

\index{semantic network}
\noindent
A set of RDF statements forms an RDF graph, where each node represents a {\em concept} which is either a subject or an object
in the RDF statement. Each edge in the graph represents a predicate. For example, a set of RDF statements shown in 
listing~\ref{lst:knowledge:rdf1} and the corresponding graphical representation is depicted in figure~\ref{fig:knowledge:rdf1}. 
The graphical representations is also called a {\em semantic network} as the graph depicts knowledge as a network of concepts,
and that the semantics of the concepts emerge from their relations depicted in the network.
 
\begin{lstlisting} [caption=Examples of simple RDF statements, label=lst:knowledge:rdf1]
Book1     instance-of   Book .
Author1   instance-of   Author .
Author2   instance-of   Author .
Book1     author        Author1 .
Book1     author        Author2 .
Book1     title         "Distributed Computing" .
\end{lstlisting}

\begin{figure}[!htbp]
	\centerline{
		\epsfig{figure=Knowledge/rdf1-g.eps, width=0.6\linewidth}
	}
	\caption{A graphical representation for the RDF statements in listing~\ref{lst:knowledge:rdf1}}
	\label{fig:knowledge:rdf1}
\end{figure}

\index{RDF graph|textbf}
\begin{definition}[RDF graph]
	An {\em RDF graph} is a set of RDF triples. 
\end{definition}

\index{RDF source|textbf}
\begin{definition} [RDF source]
	A persistent container (e.g. a file) of RDF statements is called an {\em RDF source}.
\end{definition}

\noindent
\index{N3} \index{Notation-3} \index{Turtle}
An RDF graph is a static snapshot of some information at any given point of time. Since an RDF graphs are defined as mathematical 
sets, adding or removing triples from an RDF graph yields a different RDF graph. The content of an RDF source may change over time. 
A snapshot of the contents of an RDF source can be expressed as an RDF graph. 
%
An RDF graph can be represented with various notations in an RDF source, for instance XML-based~\citep{RDF-XML:2014}, Notation 3 
(or, N3)~\citep{Berners-Lee:2011} and Turtle~\citep{Turtle:2011}. We shall follow the Turtle notation, which is compact and convenient 
for human reading and writing, in this book. 

\index{resource}
Each of the entities (subjects, predicates, or objects) in an RDF triplet is treated as an an (information) {\em resource}, and is 
uniquely identified with an IRI.

\begin{definition} [International Resource Idendifier (IRI)]
	An IRI {\em International Resource Identifier} (IRI) is a string of characters that refers to an entity in the world of 
	discourse, which is called a {\em resource}. IRI is a generalization of URI/URL and has a similar representation. 
	An IRI can be used to represent a subject, object or predicate in a RDF triplet, or an RDF source.
\end{definition}

\noindent
\index{RDF resource}
Identifying the entities in RDF with IRIs enables distributed knowledge representation. A RDF graph can be defined with resources
on different nodes on a distributed system. RDF~\citep{Decker:2000} and RDF Schema (RDFS)~\citep{RDFS:2014} provides a set of 
data modeling resources for data modeling, which are used in most of the RDF graphs. Further, several organizations have 
recommended standard vocabulary for commonly used resources, which are generally referred to in other RDF graphs~\citep{Aleman-Meza:2007}. 
Using some of such data models, we can rewrite the RDF statements in listing~\ref{lst:knowledge:rdf1} with IRI's as
in listing~\ref{lst:knowledge:rdf2}.  
%
\index{prefix!in RDF source}
Lines 1--4  in the listing shows the definitions of the prefixes, which are shorthands for the namespaces for the resources 
referred to. The resources enclosed in angular brackets ($< >$) are locally defined in the current source. Note that the concepts
``Author'' and ``Book'' and the predicate ``author'' have been defined as subclasses and subproperty of standard vocabularies defined 
in schema.org and DCMI (lines 5--7). The object in line 14 is a string literal and has been represented by enclosing it in quotes
(`` ''). In general, a literal can have different data-types, e.g. a string, an integer, a date, etc. and can be described with XML 
schema definition~\citep{XSD:2012}. To keep the examples simple, we shall consider string literals only in this book. 

\begin{lstlisting} [caption=Example of RDF statements expressed with IRIs, label=lst:knowledge:rdf2]
@prefix rdf:  <http://www.w3.org/1999/02/22-rdf-syntax-ns#> .  
@prefix rdfs: <http://www.w3.org/2000/01/rdf-schema#> .        
@prefix dc:   <http://purl.org/dc/terms/> .                    
@prefix sc:   <https://schema.org/> .                          

<#Author>    rdfs:subClassOf     sc:Person . 
<#Book>      rdfs:subClassOf     dc:BibliographicResource .
<#author>    rdfs:subPropertyOf  dc:contributor .

<#Book1>     rdf:type         <#Book> .
<#Author1>   rdf:type         <#Author> .
<#Author2>   rdf:type         <#Author> .
mb:Book1     <#author>        <#Author1> .
mb:Book1     <#author>        <#Author2> .
mb:Book1     <#title>         "Distributed Computing" .
\end{lstlisting}

\noindent
\index{anonymous resource} \index{blank node}
\index{reification}
Besides IRI and literals, there is another type of resource in RDF. An {\em anonymous resource} (also called a {\em blank node}) 
is a resource in RDF, for which an IRI is not allocated~\citep{RDF:2004}. The subject of an RDF statement can either be an IRI 
or an {\em anonymous resource}, the predicate an IRI and the object an IRI, a blank node or a literal. 

Also, note that a predicate in one statement in an RDF graph can be used as a subject 
or an object in another statement, which enables definition of new properties. For example, a new predicate ``author'' has been 
defined in line 8, and used in lines 13--14 in the example. An RDF statement can also be used as a resource, and assertions can
be made about the statement. This is known as {\it reification}.

We extend the above example and provide an example of distributed knowledge representation, where the knowledge is assumed to be encoded 
in two sources (files), say \url{http://biblio.org/mybiblio.n3} and \url{http://authors.org/myauthors.n3}. We shall use this example 
later in this chapter, and refer to the sources as {\it mybiblio} and {\it myauthors} respectively. Each of the sources contain an RDF 
graph. The RDF graph in {\it biblio} contains some information about a book, and refers to the RDF graph in {\it authors}, which 
contains information about it's authors. Both of the RDF graphs refer to external resources.

\begin{lstlisting} [caption=Example of distributed knowledge representation using RDF, label=lst:knowledge:rdf-dataset]
# -------- http://biblio.org/mybiblio.n3 -------------------------
@prefix rdf:  <http://www.w3.org/1999/02/22-rdf-syntax-ns#> .
@prefix rdfs: <http://www.w3.org/2000/01/rdf-schema#> .
@prefix dc:   <http://purl.org/dc/terms/> .                    
@prefix ma:   <http://authors.org/myauthors.n3#> .

<#Book>
    rdfs:subClassOf dc:BibliographicResource .
<#Book-1>
    rdf:type       <#Book> ;
    dc:contributor ma:Author-1 ,
                   ma:Author-2 ;
    dc:title       "Distributed Systems" ;
    dc:publisher   "IEEE_Wiley Press" ;
    dc:date        "2021" .

# -------- http://authors.org/myauthors.n3 -------------------------
@prefix rdf:  <http://www.w3.org/1999/02/22-rdf-syntax-ns#> .
@prefix rdfs: <http://www.w3.org/2000/01/rdf-schema#> .
@prefix dc:   <http://purl.org/dc/terms/> .                    
@prefix sc:   <https://schema.org/> .

<#Author>
   rdfs:subClassOf sc:Person .
<#Author-1>
   rdf:type       <#Author> ;
   rdfs:label     "RKG" ;
   sc:name        "Ratan Ghosh" ;
   sc:affiliation "IIT-Bhilai" .
<#Author-2>
   rdf:type       <#Author> ;
   rdfs:label     "HG" ;
   sc:name        "Hiranmay Ghosh" ;
   sc:affiliation "IIT-Jodhpur" .
\end{lstlisting}

\begin{figure}[!htbp]
	\centerline {
		\epsfig{figure=Knowledge/rdf-dataset, width=\linewidth}
	}
	\caption{Graphical representation of distributed knowledge described in listing~\ref{lst:knowledge:rdf-dataset}}
	\label{fig:knowledge:rdf-dataset}
\end{figure}

Considering that the knowledge incorporated in the two RDF graphs {\it myauthors} and {\it mybiblio} be possessed by two distinct
agents, distributed knowledge comprises an implication of the union of the two graphs. Note that there can be duplications, 
incompleteness and inconsistencies in the information presented in the knowledge units. We shall review an approach to resolve
such issues in section~\ref{sec:knowledge:constrained}.

\index{relational database}
\index{row-wise split} \index{column-wise split}
If we think of the data in terms of traditional relational tables (where each row describe a resource, and each column describes a 
property of a resource), both row-wise and column-wise splits are possible in an RDF description. As an example of row-wise split, 
it is possible for different organizations to maintain the list of their authors and their attributes, following their own schema. 
The bibliographic databases may also be maintained by different libraries or publishing houses.
%
As an example of column-wise split, different nodes may maintain different properties of the same resource, such as a node maintaining 
the information about the publications by an author, while another maintaining the affiliation of the authors. RDF data in real life
is generally a combination of the two types of split, and heterogeneous schema definitions on the different partitions. In effect, RDF 
provides a tremendous flexibility by allowing anybody to say anything about any resource, and to keep the information anywhere. All the 
information gets automatically integrated by references to IRIs with global scope.

\subsection{Web Ontology Language (OWL)}

The modeling primitives supported in RDF and RDFS enable creation of distributed knowledge, but lacks formalism. There are no
constraints on use of resources (subjects, predicates or objects) in any RDF statement. While it provide tremendous flexibility
to RDF for knowledge representation, it can result in inconsistencies in the knowledge-base. For example, it is possible to define 
``Book'' as a bibliographic resource, and still attribute animal-like properties (say, has a tail) to it. Thus, the representational 
primitives in RDF are inadequate for modeling the knowledge in specific domains of discourse.

\index{frame-based knowledge representation}
\index{ontology} \index{class} \index{property}
{\em Web Ontology Language} (OWL)~\citep{OWL:2004} builds on RDF and RDFS to specify constraints that enables defining a domain
model. It is possible to realize frame-based knowledge representation~\cite{Minsky:1974} using OWL. For example, 
figure~\ref{fig:knowledge:frame} depicts a frame-based representation for the knowledge presented in figure~\ref{fig:knowledge:rdf-dataset}.
The upper part (above the dashed line) of the diagram depicts a domain model, called the {\em ontology}. It comprises a set
of classes, their properties, and the relations between the classes, e.g. a book has a title, one or more authors and a publisher. 
An ontology standardizes the vocabulary and their semantics to be used in describing a domain.
The lower part of the diagram depicts a set of instances, whose properties are defined and related based on the model defined 
in the ontology.  OWL provides several advanced ontology modeling constructs, namely for defining different types of classes 
(e.g. sets and enumeration types), class operations (e.g. union and intersection of sets), and semantics of properties of the 
properties (e.g. such as transitivity, symmetry, etc.).

\begin{figure}[!htbp]
	\centerline{
		\epsfig{figure="Knowledge/frame.eps", width=\linewidth}
	}
	\caption{Frame based knowledge representation}
	\label{fig:knowledge:frame}
\end{figure}

\index{ontology|textbf}
\begin{definition}[ontology]
	An {\em ontology} defines a set of representational primitives with which to model a domain of knowledge.
	The representational primitives are typically classes (or sets), attributes (or properties), and relationships 
	(or relations among class members)~\citep{Gruber:2008}. 
\end{definition}

\index{W3C} \index{RDF graph}
Like in an RDF graph, the resources in a OWL document can be distributed over the Internet. Valid OWL documents and OWL schema 
definitions are valid RDF documents, and can be parsed with an RDF parser. An OWL parser puts additional restrictions to check 
the consistency of the statements and compliance with the schema definition. We shall not discuss further about OWL in this book, 
except for mentioning that it is the standard recommended by W3C for web-based knowledge representation. 

