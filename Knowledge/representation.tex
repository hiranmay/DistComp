\section{Knowledge Representation Techniques} 

The term ``knowledge'' has been defined in many different ways in lieterature, some of which are philosophical and some are
operational. Use of knowledge in computing demands that the knowledge be formally encoded and be processed with formal algorithms.
For our purpose, we define knowlede is an organization of data that is an abstracted representation of a domain. We consider 
knowledge to be composed of named concepts, and that the names can be used to create propositions, from which conclusions can 
be drawn. This is in compliance with the Aristotelian view of {\em Representational Theory of the Mind} (RTM)~\citep{Stanford:2020}. 
%
In this section, we present a brief review two of knowledge representation techniques that conforms to the Aristotelian view
of establishing relations among concepts to create prepositions, which when represented through formal languages, can be
reasoned with. We shall deal with the subject to the extent that enables us to appreciate the issues of distributed processing 
techniques to deal with the body of knowledge that is essentially fragmented, and partitions of which may reside over a multitude 
of processing nodes on the Internet. In-depth discussions on knowledge representation and reasoning techniques can be found in any 
standard text-book on artificial intelligence.

\subsection{Semantic Network}

\index{semantic network|textbf} \index{semantic net|see {semantic network}}
In relational models, a model of a domain is built with data related to each other. For example, a bibliographic model can be
built by relating a book with it's authors, year of publication, and such other pieces of data. A domain model, in general,
comprises many such relations. A {\em semantic network} or a {\em semantic net} provides a graphical representation for such domain 
relations. For example, figure~\ref{fig:dk:semanet} depicts a semantic network representation for the relations across a handful 
of entities. 
%
\index{concept} \index{relation}
In general, a semantic network can have any arbitrary entities (called the concepts) as the nodes, and any arbitrary relation between 
a pair of them. In it's basic form, a semantic net does not commit the semantics of the concepts and the relations. This provides 
tremendous flexibility to the representation, but also results in lack of formalism. There are many variants of semantic net, some 
of which impose constraints on the basic representation to make them more formal. A detailed discussion on the properties of different 
variants of semantic networks can be found in~\cite{Sowa:1992}.

\begin{figure}
	\centerline{
		\epsfig{figure=./Knowledge/semanet.eps,width=0.8\linewidth}
	}
	\caption{An example of a Semantic Network}
	\label{fig:dk:semanet}
\end{figure}

\subsection{Frame-based Representation}

\index{frame based knowledge representation|textbf}
While the semantic net allows arbitrary entities to be linked together with arbitrary relations, the frame-based knowledge 
representation scheme~\cite{MInsky:1974} provides a schema for structuring the relation. We explain the frame based 
representation with an example in figure~\ref{fig:dk:frame} that depicts the same entities present in figure~\ref{fig:dk:semanet}. 
%
\index{frame (knowledge representation)}
A prototype structure, called a {\em frame}, is created for a {\em class} of entities that are characterized by some common 
attributes, e.g. a book is characterized by it's author(s), title, etc.  An individual entity that constitutes a class (e.g. 
the ``Distributed Systems'' book) is called an {\em instance} of the class. A class can be a {\em subclass} of another (designated
by `is-a' relation), meaning that all instances of the former are also instances of the latter, for example, all authors are persons.

\begin{figure}[htbp!]
	\centerline{
% 		\scalebox{0.6}{\input{./Knowledge/frame.pgf}}
		\epsfig{figure=./Knowledge/frame.eps,width=0.8\linewidth}
	}
	\caption{An example frame-based knowledge representation}
	\label{fig:dk:frame}
\end{figure}

\index{slot} \index{filler}
\index{generic frame} \index{specific frame}
In the figure, the entities represented with bold letters above the dashed horizontal line represent the classes. ``Thing''
is the super-class of all the classes, meaning simply that everything is a ``thing''. The text in the boxes below the class-names
indicate their attributes, called the {\em slots}, such as a person has name as an attribute, etc. A subclass of a class inherits 
the attributes of the latter. For example, an author, being a person, will have a name. We have adopted two notations to mean 
something special.
A dotted attribute box means an optional attribute, e.g. an author may or may not have an affiliation. A `+' sign next to
an attribute indicates that their can be multiple values for the attribute, e.g. a book can have more than one author.
A frame representing a class is called a {\em generic frame}.
%
The entities represented with bold letters below the horizontal dashed line represent the entities that are instances of some
of the classes. The boxes below the instance labels (e.g. Author-1, etc.) represent their attribute values, called the {\em
fillers}. A frame representing an instance is called a {\em specific frame}. Note that an attribute value in a specific frame
can be either a literal, or another instance in the knowledge-base. 

\index{ontology|textbf} \index{property restriction}
The overlaying class definitions form a schema for the property description of the individual instances, and is referred to
as an {\em ontology} that supports modeling of a domain. For example, we have specified that a book
need to have a title, one or more authors, etc. In frame-based knowledge representation languages, it is generally possible 
to incorporate various {\em property restrictions} in the class definitions, which is an important requirement for domain modeling,
e.g. the name of a person may consist of a string of alphabetic characters only. 
