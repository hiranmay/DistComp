\vspace{2mm}
\noindent
{\bf Exercises:}

\noindent
\hgcomment{Ratan to introduce some execrcises for the earlier sections.}

\begin{enumerate}
	\item  Form table~\ref{tab:clock:Allen-transitivity}, we note that {\bf if} \texttt{A} b \texttt{B} {\bf and}
		\texttt{B} oi \texttt{C}, {\bf then} \texttt{A} (b m o di fi) \texttt{C}. Sketch relative positions
		of the three events, showing each of the possible relations between \texttt{A} and \texttt{C}.
	\item Verify the computations of the relations in the example presented in section~\ref{sec:clock:Allen-algebra}. 
		Verify semantic consistency of the results with the stated facts.
	\item In figure~\ref{fig:clock:Allen-semantics}, there are several alternatives transition paths from relation o to oi. 
		As the event boundaries are updated, the relation between two events with transit through some intermediate
		relations in each of the paths. For example, one such path is ``o $\rightarrow$ fi $\rightarrow$ di $\rightarrow$ 
		si $\rightarrow$ oi''. Enumerate all such possible paths, and sketch the event boundaries to show how the
		relation transits through these paths, for atleast two possible paths.
	\item Repeat the above exercise, using binary encoding and using pumping and pruning operations.
	\item Sketch a set of possible fuzzy membership functions $\phi_t \dots \phi_x$ that satisfy the properties mentioned
		in section~\ref{sec:clock:fuzzy}. Provide mathematical definition of the functions. You may use piece-wise
		linear functions, or combinations of sigmoid functions.
	\item Organize containment relations to depict conceptual neighborhood property. Suggest binary encoding for these
		relations so that it is possible to traverse through the neighborhood with pumping and pruning operations.
\end{enumerate}
