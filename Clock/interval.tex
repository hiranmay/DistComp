\section{Interval events}

The discussions in the preceding sections deal with ordering of events that are represented as points in time. 
The order of two point events can take three possible values, namely before, together and after. Let us represent them by 
$<$, $=$ and $>$ respectively.
%
In reality, an event has a finite duration. For example, sending of a message requires the transmission buffer to be emptied 
on a transmission channel, and receiving a message requires the transfer of incoming data on transmission channel to a 
reception buffer, each of which takes finite time. Even an audio `beep' for a time signal has a start and an end-time, 
separated by a finite, albeit small, interval. We deal with the relations between such interval events and their representations 
in this section.  We define such events as interval events as follows.

\begin{definition}[Interval event]
	An interval event \texttt{X} is an event bounded in a finite interval of time. If $s_X$ and $e_X$ represent
	the starting and ending points of the interval, the constraint $s_X < e_X$ holds.
\end{definition}

The temporal relation between two interval events \texttt{A} and \texttt{B} can be unambiguously expressed as a 4-tuple of  
the temporal relations between their endpoints: $( r(s_A, s_B), r(s_A, e_B), r(e_A, s_B), r(e_A, e_B) )$. Each member of the 
tuple can take any of the values from the set $\{<,=,>\}$. Considering the constraint $s_X < e_X$, we can have 13 valid relations
between two interval events, which are known as Allen's relations~\cite{Allen:1983}, and are summarized in 
figure~\ref{fig:clock:Allen}. 
%
In the figure, each of the relation is identified with a mnemonic, which we shall refer to in the rest of this section. The semantics 
of the relation is pictorially depicted with two horizontal bars representing the duration of a reference event \texttt{A} and 
a primary event \texttt{B}, with the assumption that time flows from left to right. The temporal relations between the end-points 
of the intervals for every event are also provided. 
% Note that it is not always necessary to specify all the members of 4-tuples to define a temporal relation between two interval 
% events. For example, in ``before'' relation, the specification $e_B < s_A$ implies the other three members of the tuple: 
% $s_B < s_A$, $s_B < e_A$ and $e_B < e_A$. 
Note that the relations 8--13 are inverse relations of 1--6, in the reverse order, e.g. 
\texttt{B} b \texttt{B} $\equiv$ \texttt{A} bi \texttt{B}. The inverse of the equality relation is itself, i.e. 
\texttt{B} eq \texttt{A} $\equiv$ \texttt{A} eq \texttt{B}.

\begin{figure}[htbp!]
\centerline{
%	\scalebox{0.8}{\input{./Clock/allen.pgf}}
	\epsfig{figure=./Clock/allen.eps,width=0.8\linewidth}
}
\caption{Allen's temporal relations.}
\label{fig:clock:Allen}
\end{figure}

\vspace{2mm}
% \noindent
% \hgcomment{Following example can be removed, if required}

\noindent
{\bf Example:}

\noindent
Consider a push-button toggle switch used to turn a light bulb on or off. The bulb is initially switched off. When the
switch is operated twice in succession, it lights on and then lights off again. Assume that switching on is triggered 
by the positive edge, i.e. the bulb turns on as soon as the switch is pressed. Switching off is triggered by the negative edge, 
i.e. the bulb turns off as soon as the switch is released. 
%
Let us denote the events representing the two operations of the switch with $\texttt{S}_1$ and $\texttt{S}_2$, and the lighting 
of the bulb as \texttt{B}. Figure~\ref{fig:clock:Allen-example} shows the relations between the three events, which can be stated 
as (a) $\texttt{S}_1$ {\bf b} $\texttt{S}_2$, (b) \texttt{B} {\bf si} $\texttt{S}_1$, and (c) \texttt{B} {\bf fi}  $\texttt{S}_2$.


\begin{figure}[htbp!]
\centerline{
%	\scalebox{0.8}{\input{./Clock/allen-example.pgf}}
	\epsfig{figure=./Clock/allen-example.eps,width=0.8\linewidth}
}
\caption{Allen's temporal relations illustrated with an example}
\label{fig:clock:Allen-example}
\end{figure}


\subsection{Allen's Interval Algebra}
\label{sec:clock:Allen-algebra}

If \texttt{A}, \texttt{B} and \texttt{C} represent three interval events, and if the temporal relations between the event-pairs 
(\texttt{A}, \texttt{B}) and (\texttt{B}, \texttt{C}) are given, it is possible to reason with them to establish some relational 
constraints between the event pair (\texttt{A}, \texttt{C}). Before proceeding with the formal algebra to deduce such constraints, 
let us provide a few intuitive examples. 
\begin{enumerate}
	\item Consider the statement ``I went to gym before having my breakfast, and I went to office after the breakfast.''
		latter. Let us represent the events ``going to gym'', ``having breakfast'' and ``going to office'' as \texttt{G}, 
		\texttt{B} and \texttt{O} respectively. The above statement can be formalized as temporal relations between the 
		event-pairs (\texttt{G}, \texttt{B}) and (\texttt{O}, \texttt{B}) as \texttt{G} b \texttt{B} and 
		\texttt{O} bi \texttt{B}. 
		%
		From these two relations, we can deduce the temporal constraint between the events \texttt{G} and \texttt{O} 
		as \texttt{G} b \texttt{O}, i.e. I went to gym before going to office. This example illustrates reasoning with 
		temporal sequencing of events.
	\item Consider the statement ``I attended office some time during yesterday, and ate my lunch when at office''. Denoting 
		the events ``yesterday'', ``attending office'' and ``eating lunch'' with \texttt{Y}, \texttt{O} and \texttt{L} 
		respectively, we can interpret the sentence as \texttt{O} d \texttt{Y} and \texttt{L} d \texttt{O}. 
		%
		From these two relations, it is possible to infer the constraint between the events \texttt{L} and \texttt{Y} 
		as \texttt{L} d \texttt{Y}, i.e. I ate lunch during yesterday. This example illustrates an hierarchical 
		decomposition of events.
	\item Consider the statement ``Ram came in the room while Shyam was there, and Jadu came into the room when Ram
		was there''. Representing the three events of Ram's, Shyam's and Jadu's being in the room as \texttt{R},
		\texttt{S}, and \texttt{J}, we can formalize the statement as \texttt{S} o \texttt{R} and \texttt{J} o \texttt{R}.
		%
		In this case, we have three distinct possibilities: 
		\begin{itemize}
			\item Shyam left the room before Jadu came in, i.e. \texttt{S} b \texttt{J}, 
			\item Shyam and Jadu crossed their ways at the door, i.e. \texttt{S} m \texttt{J}, and 
			\item Shyam and Jadu were together in the room for some time, i.e. \texttt{S} o \texttt{J}. 
		\end{itemize}
		Note that while the relational constraints between \texttt{S} and \texttt{J} can be uniquely resolved to a single
		relation in the earlier examples, it is the disjunction of three relations in this example. We shall use the notation 
		\texttt{C} (b m o) \texttt{A} to represent the same.
\end{enumerate}


With this intuitive understanding, we proceed with a formal presentation of Allen's interval algebra. In the following treatment, we 
consider three events as $E_i$, $E_j$ and $E_k$, and we represent the relational constraint between the events as $R_{ij}$, $R_{jk}$ 
and $R_{ik}$ respectively. In general, a relational constraint is a disjunction of some of the Allen's relations. 
%
Given a relation $p \in R_{ij}$ and $q \in R_{jk}$, a {\em transitivity table} as shown in table~\ref{tab:clock:Allen-transitivity}, 
is used to derive the resulting constraint on the relation $R_{ik}$. The header row and the column in the table 
represent the possible relations in $R_{ij}$ and in $R_{jk}$ respectively. An entry in the table, designated as $T(p,q)$, where $p$ 
is the row number and $q$ is the column number, represents the relational constraints imposed on $R_{ik}$ for the corresponding values 
$p \in R_{ij}$ and $q \in R_{jk}$. When there are more than one relation specified in a table cell, it means that the imposed constraint
is a logical disjunction of those relations. For example, the entry $T(\texttt{di},\texttt{b}) = (\texttt{b m o fi di})$ implies that 
the constraint imposed
by relations $\texttt{di} \in R_{ij}$ and $\texttt{b} \in R_{jk}$ on $R_{ik}$ allows the latter to assume any of the five values. Some 
of the cells are marked with $\mathbb{NC}$, meaning that there will be no constraint imposed on $R_{ik}$ for the corresponding values 
of $(p,q)$, i.e. it is a disjunction of all of 13 Allen's relations. The reader is advised to check the validity of at least some of the 
entries of the table.
 
The constraint $C$ imposed on $R_{ik}$ by $R_{ij}$ and $R_{jk}$ is computed as a logical disjunction for all possible combinations of 
$T(p,q) \forall p \in R_{ij}, \forall q \in R_{jk}$ as shown in algorithm~\ref{algo:clock:constraint}. Note that $C$ is the additional 
constraint on $R_{ik}$ imposed by the constraints $R_{ij}$ and $R_{jk}$ and is not it's final value. 

\begin{algorithm}[!htbp]
	\SetAlgoLined
	\SetKwProg{proc}{procedure}{}{end}
	\DontPrintSemicolon
	\proc{Constraint($R_{ij}, R_{jk}$)}{
		$C=\emptyset$\;
		\For{each $p \in R_{ij}$}{
    			\For{each $q \in R_{jk}$}{
        			$C \leftarrow C \cup T(p,q)$\;
    			} % EndFor
		} % EndFor
		\Return $C$\;
	} % EndProc
	\caption{Computing relational constraint}
	\label{algo:clock:constraint}
\end{algorithm}

\begin{table}[htbp!]
	\tiny
	\caption{Transitivity table for temporal relations \label{tab:clock:Allen-transitivity}} {

	% \begin{center}
		% \begin{tabular}{|p{8mm}||p{8mm}|p{8mm}|p{8mm}|p{8mm}|p{8mm}|p{8mm}|p{8mm}|p{8mm}|p{8mm}|p{8mm}|p{8mm}|p{8mm}|p{8mm}|}
		\begin{tabular}{|p{3mm}||p{5mm}|p{5mm}|p{5mm}|p{5mm}|p{5mm}|p{5mm}|p{5mm}|p{5mm}|p{5mm}|p{5mm}|p{5mm}|p{5mm}|p{5mm}|}
		\hline
		    & {\bf b} & {\bf bi} & {\bf d} & {\bf di} & {\bf o} & {\bf oi} & {\bf m} & {\bf mi} & {\bf s} & {\bf si} & {\bf f} & {\bf fi} & {\bf eq }\\ 
		\hline
		\hline
			{\bf b} & b & $\mathbb{NC}$ & b m o s d & b & b & b m o s d & b & b m o s d  & b & b & b m o s d & b & b \\
		\hline
			{\bf bi} & $\mathbb{NC}$  & bi & bi mi oi f d & bi & bi mi oi f d & bi  & bi mi oi f d & bi & bi mi oi f d & bi & bi & bi & bi \\
		\hline
			{\bf d} & b & bi & d & $\mathbb{NC}$  & b m o s d & bi mi oi f d & b & bi & d & bi mi oi f d & d & b m o m s d & d \\
		\hline
		{\bf di} & b m o fi di & bi mi oi si di & o fi di si eq s d f oi & di & o fi di & oi si di & o fi di & oi si di & o fi di & di & oi si di & di & di \\
		\hline
		{\bf o} & b & bi mi oi si di & o s d & b m o fi di & b m o & o fi di si eq s d f oi & b & oi si di & o & di fi o & d s o & b m o & o \\
		\hline
		{\bf oi} & b m o fi di & bi & oi f d & bi mi oi si di & o fi di si eq s d f oi & bi mi oi & o fi di & bi & oi f d & oi mi bi &oi & oi si di & oi \\
		\hline
		{\bf m} & b & bi mi oi si di & o s d & b & b & o s d & b & f eq fi & m & m & d s o & b & m \\
		\hline
		{\bf mi} & b m o fi di & bi & oi f d & bi & oi f d & bi & s eq si & bi & oi f d & bi & mi & mi & mi \\
		\hline
		{\bf s} & b & bi & d & b m o fi di & b m o & oi f d & b & mi & s & s eq si & d & b m o & s \\
		\hline
		{\bf si} & b m o fi di & bi & oi f d & di & o fi di & oi & o fi di & mi & s eq si & si & oi & di & si \\
		\hline
		{\bf f} & b & bi & d & bi mi oi si di & o s d & bi mi oi & m & bi & d & bi mi oi & f & f eq fi & f \\
		\hline
		{\bf fi} & b & bi mi oi si di & o s d & di & o & oi si di & m & oi si di & o & di & f eq fi & fi & fi \\
		\hline
		{\bf eq} & b & bi & d & di & o & oi & m & mi & s & si & f & fi & eq \\ 
		\hline
	\end{tabular} 
	} {}
	% \end{center}
	% \caption{Transitivity table for temporal relations}
	% \label{tab:clock:Allen-transitivity}
\end{table}

% \hgcomment{
% 	A note on idempotence: If you look at the diagonal of the table, you will find some idempotent relations besides eq, e.g. 
% 	$T(\texttt{b,b}) = \texttt{b}$, etc. But, I am not sure if they worth a mention without some discussions on their semantics.
% 	Need to think it through.
% }

In general, when we deal with a number of events, they can be organized in a network, where the nodes represent the events and the 
edges represent their relational constraints. We refer to the network as an {\em event network}. It is assumed that the network maintains 
the complete information about the relational constraints between the events that are known at any point of time. When a new relational 
constraint is added, all its consequences are computed recursively in the network as a transitive closure of the new constraint and the 
existing ones, as shown in algorithm~\ref{algo:clock:add}. 
%
% \noindent
A brief explanation of the algorithm is as follows. The superscripts $old$ and $new$ refer to the old (existing) and the new 
values of a constraint respectively. They are used as temporarily variables for comparison purposes, to decide if further recursion 
is needed. $Q$ maintains a queue of such new constraints, either externally specified, or generated at the intermediate stages of the 
recursive computation, for which the consequences are yet to be determined. The functions {\em Comparable$(i,k)$} and {\em Comparable$(k,j)$}
yield the set of all nodes $k$ in the network, for which the constraints $R_{ik}$ and $R_{kj}$ respectively are likely to be impacted by 
a change in $R_{ij}$. The function {\em Constraint$(\circ, \circ)$} is computed with algorithm~\ref{algo:clock:constraint}.
%
Addition of a new constraint in the network restricts the existing constraints further. The updated constraint between a pair of 
events is the intersection of the existing constraints and the new ones resulting from the changes in other parts of the network.
If this intersection is an empty set, it implies the impossibility of the situation.
The recursive computation continues as long as there is something to update. Since there are 13 possible relations, the worst case 
complexity of the algorithm in a network with $N$ nodes is $13 \times \frac{(N-1)(N-2)}{2}$.

\begin{algorithm}[!htpb]
	\SetAlgoLined
	\SetKwProg{proc}{procedure}{}{end}
	\DontPrintSemicolon
	\proc{Add($R^{new}_{ij}$)}{
		add $R^{new}_{ij}$ to queue $Q$\; 
		\While {$Q$ is not empty}{
			get next $R^{new}_{ij}$ from $Q$\;
			$R^{old}_{ij} \leftarrow R_{ij}$\;
			$R_{ij} \leftarrow R_{ij} \cap R^{new}_{ij}$\;
			\If {$R_{ij} \subset R^{old}_{ij}$} {
				\For {each node $k$, such that Comparable$(k,j)$} {
					$R^{old}_{kj} \leftarrow R_{kj}$\;
					$R_{kj} \leftarrow R_{kj} \cap$ Constraint$(R_{ki},R_{ij})$\; 
					\If {$R_{kj} \subset R^{old}_{kj}$} {
						add $R_{kj}$ to queue $Q$\;
					} % EndIf
				} % EndFor
				\For {each node $k$, such that Comparable$(i,k)$} {
					$R^{old}_{ik} \leftarrow R_{ik}$\;
					$R_{ik} \leftarrow R_{ik} \cap$ Constraint$(R_{ij},R_{jk})$\; 
					\If {$R_{ik} \subset R^{old}_{ik}$} {
						add $R^n_{ik}$ to queue $Q$\;
					} % EndIf
				} % EndFor
			} % EndIf
		} % EndWhile
		\Return void\;
	} % EndProc
	\caption{Addition of a new constraint in event network}
	\label{algo:clock:add}
\end{algorithm}

\vspace{2mm}
\noindent
{\bf Example:}

\noindent
Consider the statement: ``I clicked the switch to turn on the light''. It establishes a temporal constraint between two events 
clicking the switch, and the light being on. Designating the two events with \texttt{S} and \texttt{L}, the temporal constraint
between the events can be expressed as $R_{SL} \equiv$ \texttt{S} (m o) \texttt{L}. 
Now, consider that we add a statement ``Rajan was not there in the room when I clicked the switch''. We have introduced a new event
Rajan's presence in the room. Denoting the event with \texttt{R}, it's relation with clicking the switch can be formalized as 
$R_{RS} \equiv$ \texttt{R} (b m mi bi) \texttt{S}. 
The event network depicting the three events and the {\it specified temporal} constraints is shown in 
figure~\ref{fig:clock:Allen-algebra-example}(a).

\begin{figure}[htbp!]
        \subfigure[]{
                \begin{minipage}{\linewidth}
                \centerline{
%			\scalebox{0.8}{\input{./Clock/allen-algo-1.pgf}}
			\epsfig{figure=./Clock/allen-algo-1.eps,width=0.4\linewidth}
                }
                \end{minipage}
        }
        \subfigure[]{
                \begin{minipage}{0.5\linewidth}
                \centerline{
%			\scalebox{0.8}{\input{./Clock/allen-algo-2.pgf}}
			\epsfig{figure=./Clock/allen-algo-2.eps,width=0.8\linewidth}
                }
                \end{minipage}
        }
        \subfigure[]{
                \begin{minipage}{0.5\linewidth}
                \centerline{
%			\scalebox{0.8}{\input{./Clock/allen-algo-3.pgf}}
			\epsfig{figure=./Clock/allen-algo-3.eps,width=0.8\linewidth}
                }
                \end{minipage}
        }
	\caption{Stages of event network update}
	\label{fig:clock:Allen-algebra-example}
\end{figure}

Addition of the second relation prompts a computation for the relation $R_{RL}$ between the events \texttt{R} and \texttt{L}, i.e. 
Rajan's presence in the room and the light being on. We find $R_{RL}$ to be \texttt{R} (b si eq s d f oi mi bi) \texttt{L}, using 
algorithm~\ref{algo:clock:constraint} with $R_{RS}$ and $R_{SL}$ as the parameters. 
% To compute the constraint, we note that
% 
% \begin{tabbing}
% 1234\=1234567890123456\=1234567890123456\=12345678901234567890\=123456789012345678\=\kill
% 	\> $T$(b,m) = b \> $T$(m,m) = b  \> $T$(mi,m) = (s eq si) \> $T$(bi,m) = (bi mi oi f d) \\
% 	\> $T$(b,o) = b \> $T$(m,o) = b  \> $T$(mi,o) = (oi f d) \> $T$(bi,o) = (bi mi oi f d) \\
% \end{tabbing}
% 
% \noindent
% We compute $R_{RL}$ as the union of all these relations, i.e. \texttt{R} (b si eq s d f oi mi bi) \texttt{L}.
The updated network is shown in figure~\ref{fig:clock:Allen-algebra-example}(b). 
% Note that this disjunction includes all possible relations, except m, o, fi and di. 
The reader is advised to check the semantic consistency of this outcome with the stated facts.

Let us assume that we get some further information, namely ``Rajan was there in the room at a later time while the light was 
on''. This new information can be formalized as an added constraint $R^{new}_{RL} \equiv$ \texttt{R} (oi si di) \texttt{L}. Inclusion
of the new constraint results in an update in the value of $R_{RL}$ as \texttt{R} (oi si) \texttt{L}, the intersection of the 
existing constraint and the new one. The update in $R_{RL}$ triggers a recursive update in the network that can be computed with 
algorithm~\ref{algo:clock:add}. 
%
A change in $R_{RS}$ will be effected using constraints $R_{RL}$ (updated) and $R_{LS}$~\footnote{The constraint $R_{LS}$ consists of 
the inverse of the relations in the constraint $R_{SL}$, i.e. \texttt{L} (mi oi) \texttt{S}}. We can find $R^{new}_{RS}$ as \texttt{R} 
(bi mi oi) \texttt{S} using algorithm~\ref{algo:clock:constraint}. Taking its intersection with its existing value, we get $R_{RS}$ as 
\texttt{R} (bi mi) \texttt{S}. The recursion stops here, since there is no more nodes to traverse. 
The updated network is shown in figure~\ref{fig:clock:Allen-algebra-example}(c). 
 
The semantics of change in $R_{RS}$ is as follows. The new knowledge that Rajan was there later in the room, combined with
the earlier knowledge that he was not there when the switch was operated, excludes the possibility of his being in the room
before the switch was clicked. Of course, it is assumed that the event of Rajan's presence in the room is one single interval, 
and we discount the possibility of his leaving the room temporarily when the switch was operated.
This example illustrates the practical utility of Allen's interval algebra in reasoning with the temporal relations
of interval events. The reader is advised to verify the computations in this example and the semantic consistency of the 
results with the stated facts.

\subsection{Conceptual neighborhood}

The discussions in the earlier sections of this chapter show that despite various clock synchronization methods, the sequence of 
point events, recorded by different processors on a distributed system, cannot always be precisely determined. For example, if the 
audio of a gun-shot and the visual scene of a man falling are recorded by a microphone and a video camera connected to different 
processors in a distributed system, it may be difficult to establish the exact sequence of the start and the end points of the 
audio and the visual events. There are many factors other than network delays in a distributed system, which also contributes to 
the uncertainty, for example delays introduced by the recording equipment, the difference in the speeds of light and sound, 
and inherent nondeterminism in start and end points of an event~\footnote{For example, when does a man, who is already in motion 
(say walking), start falling? when does he stop falling?}. Moreover, in the last section, we have seen that the inferred relation 
between event pairs cannot always be unambiguously resolved. Thus, there are many situations when we may have to be satisfied with 
the inexact knowledge about the relations between interval events.

  
\begin{figure}[htbp!]
	\centerline{
%		\scalebox{0.5}{\input{./Clock/allen-semantics.pgf}}
		\epsfig{figure=./Clock/allen-semantics.eps,width=0.8\linewidth}
	}
	\caption{Semantics of Allen's relations}
	\label{fig:clock:Allen-semantics}
\end{figure}
 
The ambiguity in establishing the temporal relations between the events motivates exploration of the nature of the inexactness
in the relations. Allen's relations can be placed on a on a two dimensional grid~\cite{Freska:1992} as shown in
figure~\ref{fig:clock:Allen-semantics}, to understand their semantics. The edges of the grid are marked with the comparison between 
the start and the end points of the two events, designated as \texttt{A} and \texttt{B}. The 13 relations are depicted as cells in 
the grid in their respective positions for the comparison results. We observe that only one of the comparison parameters change 
between any two adjacent relations, vertically, horizontally, or diagonally.
It implies that it is possible to move from one relation to any of its adjacent relation by continuously deforming (i.e. by 
continuously changing the end-points of) either or both the events without satisfying any other relation at an intermediate stage. 
Thus, a small error in perceiving the end-points of the events is likely to result in misclassification of their relation to 
an adjacent one.

The adjacent relations in figure~\ref{fig:clock:Allen-semantics} are called {\em conceptual neighbors}. A set of relations forms 
a {\em conceptual neighborhood}, if the its members are connected through conceptual neighborhood relations (e.g. {\bf b}, {\bf m} and 
{\bf o}).  Note that the members of any of the entries in the transitivity table used in Allen's interval algebra 
(table~\ref{tab:clock:Allen-transitivity}) belong to a conceptual neighborhood (and that it is not an accident). 
%
Thus, when the relation between two events are not known with certainty, it is still possible to represent their relation in an 
inexact fashion as a conceptual neighborhood. Such inexact descriptions are often useful for logical reasoning with interval events, 
as illustrated in an earlier example.

\subsection{Binary encoding}

A computationally efficient binary encoding scheme for reasoning with the relations between two interval events has been proposed 
in~\cite{Papadias:2001}. In this scheme, the relation of a primary event \texttt{B} is established with a reference event \texttt{A}
in the following way. We define five neighborhoods of the reference event \texttt{A} as follows:

\begin{itemize}
	\item \texttt{t}: $(-\infty, s_A)$, the open interval from the beginning of time to the starting point of \texttt{A}, excluding 
		the starting point.
	\item \texttt{u}: $[s_A, s_A]$, the infinitesimally small closed interval covering the starting point of \texttt{A}.
	\item \texttt{v}: $(s_A, e_A)$, the open interval covering the duration of \texttt{A}, excluding the endpoints.
	\item \texttt{w}: $[e_A, e_A]$, the infinitesimally small closed interval covering the ending point of \texttt{A}.
	\item \texttt{x}: $(e_A, \infty)$, the open interval from the ending point of \texttt{A} to the end of time, excluding 
		the ending point.
\end{itemize}

\begin{figure}[htbp!]
	\centerline{
%		\scalebox{0.8}{\input{./Clock/papadias.pgf}}
		\epsfig{figure=./Clock/papadias.eps,width=0.8\linewidth}
	}
	\caption{Binary encoding of temporal relations}
	\label{fig:clock:Papadias}
\end{figure}
 
\noindent
The neighborhoods are shown in figure~\ref{fig:clock:Papadias}. The relation of \texttt{B} with \texttt{A} is represented
with 5 bits, the truth value of each representing the intersections of the interval of \texttt{B} with one of the neighborhoods of 
\texttt{A}. For example, in figure~\ref{fig:clock:Papadias}, the relationship of \texttt{B} with \texttt{A} can be represented
by 11100, since the interval of \texttt{B} intersects with the neighborhoods \texttt{t}, \texttt{u} and \texttt{v} of \texttt{A}.
It is evident that the different temporal relations between two events can be represented by continuous set of 1s in the bit-string
(excluding 01000 and 00010, since events of infinitesimally small durations are discounted). Indeed, there are exactly 13 valid
bit-strings in this representation, each corresponding to one of the Allen's relations as shown in table~\ref{tab:clock:Papadias}.

\begin{table}
	\footnotesize
	\caption{Binary encoding of temporal relations \label{tab:clock:Papadias}} {
	% \begin{center}
		\begin{tabular}{p{10mm} p{10mm} |   p{10mm} p{10mm} |  p{10mm} p{10mm} | p{10mm} p{10 mm}} 
		\hline
		Bit-stream & Allen Relation & Bit-stream & Allen Relation & Bit-stream & Allen Relation 
			& Bit-stream & Allen Relation \\ 
		\hline
		10000 & \texttt{B} b  \texttt {A} & 11000 & \texttt{B} m  \texttt {A} & 11100 & \texttt{B} o  \texttt {A} 
			& 11110 & \texttt{B} fi \texttt {A} \\
		11111 & \texttt{B} di \texttt {A} & 01111 & \texttt{B} si \texttt {A} &  01110 & \texttt{B} eq \texttt {A} 
			& 01100 & \texttt{B} s  \texttt {A} \\
		00100 & \texttt{B} d \texttt {A} & 00110 & \texttt{B} f  \texttt {A}  & 00111 & \texttt{B} oi \texttt {A} 
			& 00011 & \texttt{B} mi \texttt {A}  \\
		00001 & \texttt{B} bi \texttt {A} & & & & & & \\
		\hline
		\end{tabular}
	}{}
	% \end{center}
	% \caption{Binary encoding of temporal relations}
	% \label{tab:clock:Papadias}
\end{table}

There is a simple relation between binary encoding and conceptual neighborhood shown in figure~\ref{fig:clock:Allen-semantics}. 
It is easy to see that one can move to a neighboring relation by ``pumping'' or ``pruning'' an 1 on either side of a bit-stream encoding.
For example pumping an 1 to the right of ``meets'' (11000) takes it to ``overlaps'' (11100), and further pruning an 1 to the left takes 
it to ``starts'' (01100).
% 
% $11000 \xrightarrow{pump-right} 11100 \xrightarrow{prune-left} 01100$
%
The number of bits to be pumped or pruned to reach one relation from another can be considered a metric for the semantic distance 
(in the space of conceptual neighborhood) between two relations. For example, the semantic distance between ``meets'' and ``starts'' 
is 2.
 
An efficient algorithm for computing the semantic distance between two bit-encoded relations $R_1$ and $R_2$ is presented in 
algorithm~\ref{algo:clock:semantic-distance-binary}. This is useful for approximate search in event databases, like find the events
that {\it approximately} overlaps with a gun-shot being heard. Given a reference event $\texttt{E}_r$ and a reference relation 
$R_r$, the query can be formalized as ``fetch all events $\texttt{E}$ such that 
$d( R(\texttt{E}, \texttt{R}_r), R(\texttt{E}, \texttt{R}_r)) \leq \tau$, where $d$ is the semantic distance between the relations, and 
$\tau$ is an arbitrary constant. The idea is to implement a neighborhood of radius $\tau$ in the relation space around the
reference event and the reference query, and to fetch all events whose relations with the reference event that fall in that neighborhood. 
It is a useful tool when the relation between event endpoints captured on different nodes of a distributed system cannot be ascertained.
We shall not discuss the database design issues to implement such queries in this chapter.

\begin{algorithm}[!htbp]
	\SetAlgoLined
	\DontPrintSemicolon
	\SetKwProg{proc}{procedure}{}{end}
	\proc{SemanticDistance($R_1, R_2$)}{
		$R = R_1$ \texttt{bit-or} $R_2$ \;
		$i_{left}, i_{right} =$ left-most, right-most bit-position of $R$ that is 1\;
		$d = 0$\;
		\For{$i = i_{left}$  to $i = i_{right}$}{
			\If {$R_1[i] = 0$} {
				$d \leftarrow d+1$\;
			} % EndIf
			\If {$R_2[i] = 0$} {
				$d \leftarrow d+1$\;
			} % EndIf
		} % EndFor
		\Return $d$\;
	} % EndProc
	\caption{Computing semantic distance for bit-encoded temporal relations}
	\label{algo:clock:semantic-distance-binary}
\end{algorithm}

\subsection{Fuzzy representation}
\label{sec:clock:fuzzy}

Papadias'es binary encoding scheme, discussed in the last section, enables comparing the relations between event-pairs, 
but with a large granularity. It is possible to define the scheme at a more detailed level by adding more neighborhoods 
and using more number of bits to represent the relations. For example, with introduction of two additional neighborhoods 
with duration $\delta$ before the staring point and after the ending point of \texttt{A}, and using a 7-bit representation, 
we can distinguish between ``far-before'', ``near-before'', ``near-after'' and ``far-after''. Nevertheless, such schemes
with any finite number of intervals will still have granularity, and the choice of neighborhood definitions may seem arbitrary.
%
To overcome this difficulty, \cite{Wattamwar:2008} proposes a representation for relation between interval events that is 
essentially a generalization of Papadias'es binary encoding scheme. In this approach, a relation is expressed as a five-tuple
of numbers ($\mu_t, \mu_u, \mu_v, \mu_w, \mu_x$), (in place of 5 bits) each of which represent a fuzzy membership function for 
the primary event to intersect with a neighborhood of the reference event. 

\begin{figure}[htbp!]
	\centerline{
%		\scalebox{0.8}{\input{./Clock/fuzzy.pgf}}
		\epsfig{figure=./Clock/fuzzy.eps,width=0.8\linewidth}
	}
	\caption{Fuzzy encoding of temporal relations}
	\label{fig:clock:Wattamwar}
\end{figure}


In order to determine the fuzzy membership value for the neighborhoods, a fuzzy membership function is associated with each of 
the neighborhoods of the reference event. Let them be designated as $\phi_t$, $\phi_u$, $\phi_v$, $\phi_w$ and $\phi_x$ respectively. 
%
While the functions can be designed in different ways, a general property for them is as follows. A function should attain the maximum
value of 1 in it's own neighborhood and 0 away from it, with graded transition at either end. We depict a piece-wise linear 
function for one of them ($\phi_v$) satisfying this property in figure~\ref{fig:clock:Wattamwar}. 
%
The maximum value of the fuzzy function attained in the interval represented by the primary event determines the corresponding 
fuzzy membership value, for the relation. For example, in figure~\ref{fig:clock:Wattamwar}, we note that the maximum value attained 
by $\phi_v$ in the interval represented by event \texttt{B} is 0.8, making it the value of $\mu_v$ for the relation of event \texttt{B} 
with respect to event \texttt{A}. With reasonable assumptions about the shape of the other fuzzy membership functions, the
relation of \texttt{B} with respect to \texttt{A} in the diagram can be expressed as $(1.0, 1.0, 0.8, 0.0, 0.0)$. While crisp
``meet'' and overlap relations are expressed as $(1,1,0,0,0)$ and $(1,1,1,0,0)$ respectively, the fuzzy representation suggests
that the relation between the two events is known with some ambiguity, with an assumption that either or both of $e_B$ and $s_A$
are not accurately known. 
 
The fuzzy representation can be used to determine a graded distance between two relations for implementing approximate
relational queries. The algorithm for computation of semantic distance for fuzzy relations is as shown in 
algorithm~\ref{algo:clock:semantic-distance-fuzzy}. 


\begin{algorithm}[!htbp]
	\SetAlgoLined
	\DontPrintSemicolon
	\SetKwProg{proc}{procedure}{}{end}
	\proc{SemanticDistance($R_1[5], R_2[5]$)} {
		\For {$i = 1 .. 5$}{
			$R[i] = \texttt{max}(R_1[i],R_2[i])$\; 
		} % EndFor
		$i_{min}, i_{max} =$ minimum, maximum value of $i$ for which $R[i] \neq 0$\;
		\For {$i = i_{min} .. i_{max}$}{
			\eIf {$R[i] = 0$} {
				$d \leftarrow d + 1$\;
			}
			{
				$d \leftarrow d + (R[i] - R_1[i])$ \; 
				$d \leftarrow d + (R[i] - R_2[i])$ \; 
	
			} % EndIf
		} % EndFor
		\Return $d$\;
	} % EndProc
	\caption{Computing semantic distance for fuzzy temporal relations}
	\label{algo:clock:semantic-distance-fuzzy}
\end{algorithm}

\subsection{Spatial Events}

The events dealt with a distributed system need not always be restricted to temporal dimension alone. For example, visual events
as captured with video-cameras cover some spatial extents as well. Allen's interval algebra and its extensions in the forms of
bit-encoded or fuzzy representations, are equally applicable to any spatial dimension. The semantics of the relations need to be 
appropriately modified in such cases, for example, the relation \texttt{b} may stand for ``to the left of'' or ``above''. 
%
As in the case of temporal dimension, there may be ambiguities in ascertaining the spatial relation between the end-points of
the events, as exemplified in the differences found in relative positions of objects seen with two different cameras, separated
in space. This necessitates the need for considering conceptual neighborhoods and approximate queries with spatial events as well.

It may seem to be logical to represent relations between multi-dimensional events as tuples of elementary relations, each of which 
refers to the relation of the projections of the events on a specific dimension. For example, a visual event in a video is restricted 
to two dimensional space ($x,y$) and time  ($t$), and one may be tempted to represent the relation between such events as 
($R^X, R^Y, R^T$), where each of the $R$s can be one of the Allen's relation. But, it must  be appreciated that all discussions on 
interval events presented in the earlier sections are valid so long there is a strict ordering relation between the event endpoints. 
With two or more spatio-temporal dimensions, the orderings between the event endpoints become partial, and ambiguities creep in. For 
example, it is easy to see that the two distinct relations presented in figures~\ref{fig:clock:Allen-ambiguity}(a) and (b) cannot 
be distinguished with Allen's relations in the projected spaces alone.


\begin{figure}[htbp!]
	\subfigure[\texttt{B} is contained in \texttt{A}]{
                \begin{minipage}{0.5\linewidth}
                \centerline{
%			\scalebox{0.5}{\input{./Clock/Allen-ambiguity-1.pgf}}
			\epsfig{figure=./Clock/allen-ambiguity-1.eps,width=0.8\linewidth}
                }
                \end{minipage}
        }
	\subfigure[\texttt{B} is outside \texttt{A}]{
                \begin{minipage}{0.5\linewidth}
                \centerline{
%			\scalebox{0.5}{\input{./Clock/Allen-ambiguity-2.pgf}}
			\epsfig{figure=./Clock/allen-ambiguity-2.eps,width=0.8\linewidth}
                }
                \end{minipage}
        }
	\caption{Ambiguity in Allen's relations when extended to multi-dimensional space}
	\label{fig:clock:Allen-ambiguity}
\end{figure}

The ambiguity can be resolved with introduction of one more set of relations that specifies the of intersection
between two events. These containment relations are shown in figure~\ref{fig:clock:containment}. These relations
can also have bit-string or fuzzy encodings as the Allen's relation. Thus, the relation between two events in
a multi-dimensional space of dimension $N$ can be specified with an $N+1$-tuple, where one of the members represent
the relation in containment space and each of the other $N$ members specify an Allen's relation for the  projections
of the events on one of the dimensions.

\begin{figure}[htbp!]
	\centerline{
%		\scalebox{0.8}{\input{./Clock/containment.pgf}}
		\epsfig{figure=./Clock/containment.eps,width=0.8\linewidth}
	}
	\caption{Containment relations}
	\label{fig:clock:containment}
\end{figure}
