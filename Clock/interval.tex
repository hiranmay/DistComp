\section{Interval events}

The discussions in the preceding sections deal with ordering of events that are represented as points in time. 
The order of two point events can take three possible values, namely before, together and after. Let us represent them by 
$<$, $=$ and $>$ respectively.
%
In reality, an event has a finite duration. For example, sending of a message requires the transmission buffer to be emptied 
on a transmission channel, and receiving a message requires the transfer of incoming data on transmission channel to a 
reception buffer, each of which takes finite time. Even an audio `beep' has a start and an end-time, 
separated by a finite, albeit small, interval. We deal with the relations between such interval events and their representations 
in this section.  We define such events as interval events as follows.

\index{interval event|textbf}
\begin{definition}[Interval event]
	An interval event \texttt{X} is an event bounded in a finite interval of time. If $s_X$ and $e_X$ represent
	the start and end points of the interval, then $s_X < e_X$.
\end{definition}


\begin{figure}[htbp!]
\centerline{
	\epsfig{figure=./Clock/allen.eps,width=\linewidth}
}
\caption{Allen's temporal relations.}
\label{fig:clock:Allen}
\end{figure}

\index{Allen's relations}
The temporal relation $r_{AB}$ between two interval events \texttt{A} and \texttt{B} can be unambiguously expressed as a 4-tuple of  
the temporal relations between their endpoints. 
\begin{equation}
	r_{AB} \equiv ( r(s_A, s_B), r(s_A, e_B), r(e_A, s_B), r(e_A, e_B) ). 
\end{equation}
where each member of the tuple can take any of the values from the set $\{<,=,>\}$. 
Considering the constraint $s_X < e_X$, we can have 13 valid relations between two interval events, which are known as Allen's 
relations~\citep{Allen:1983}, and are summarized in figure~\ref{fig:clock:Allen}. 
%
In the figure, each of the relation is identified with a mnemonic, which we shall refer to in the rest of this section. The semantics 
of the relation is pictorially depicted with two horizontal bars representing the duration of a reference event \texttt{A} and 
a primary event \texttt{B}, with the assumption that time flows from left to right. The temporal relations between the end-points 
of the intervals for every event are also provided. 
% Note that it is not always necessary to specify all the members of 4-tuples to define a temporal relation between two interval 
% events. For example, in ``before'' relation, the specification $e_B < s_A$ implies the other three members of the tuple: 
% $s_B < s_A$, $s_B < e_A$ and $e_B < e_A$. 
Note that the relations 8--13 are inverse relations of 1--6, in the reverse order, e.g. 
\texttt{B} b \texttt{A} $\equiv$ \texttt{A} bi \texttt{B}. The inverse of the equality relation is itself, i.e. 
\texttt{B} eq \texttt{A} $\equiv$ \texttt{A} eq \texttt{B}.

\subsection{Allen's Interval Algebra}
\label{sec:clock:Allen-algebra}

\index{interval algebra|see {Allen's interval algebra}}
\index{Allen's interval algebra}
Consider three interval events $E_i$, $E_j$ and $E_k$. Let the relational constraints between the events be denoted
by $R_{ij}$, $R_{jk}$ and $R_{ik}$ respectively.  The relational constraints between any two pair of events imply some relational 
constraint between the third pair. For example, 
$(E_i \text{ b } E_j) \text{ and } (E_k \text{ bi } E_j) \rightarrow (E_i \text{ b } E_k)$.
Allen's interval algebra formalizes such implications on the relational constraints.
 
In general, the relational constraints between any two events are represented by a disjunction of a set of Allen's relation. For 
example, The statement ``$E_i$ occurs either before $E_j$, or after $E_j$'' can be encoded as $R_{ij} \equiv (\text{b bi})$.  
We use $R_{ij}$ to represent the set $\{\text{b, bi}\}$, and the logical disjunction $(\text{b} \vee \text{bi})$ interchangeably,
the meaning being clear from the context. 

\noindent
\index{transitivity function}
\index{transitivity table}
Given a relation $p \in R_{ij}$ and $q \in R_{jk}$, we define a {\em transitivity function} $T(p,q) \rightarrow S$ to determine
$S$, the implied constraint on the relation $R_{ik}$. In general, $S$ is a disjunction of a set of Allen's relation. For example,
$T(\text{b,d}) \rightarrow (\text{b m o s d})$, which the reader is advised to verify. The set of all transitivity functions makes
a {\em transitivity table} (the complete table is available in~\citep{Allen:1983}), which defines the mapping
\begin{equation}
	\mathbb{T}: \mathcal{A} \times \mathcal{A} \rightarrow \{ \mathcal{A} \}
\end{equation}

\noindent
The relational constraint $C$ imposed on $R_{ik}$ by the constraints $R_{ij}$ and $R_{jk}$ is given by
\begin{equation}
	C = \cup_{\forall p \in R_{ij}, q \in R_{jk}} T(p,q)
\label{eqn:clock:implication}
\end{equation}

\noindent
Note that $C$ is the additional constraint on $R_{ik}$ imposed by the constraints $R_{ij}$ and $R_{jk}$ and is not it's final value. 
There may be some constraints \`{a}-priori specified for $R_{ik}$. The final value of of $R_{ik}$ is given by
\begin{equation}
	R_{ik} \rightarrow R_{ik} \cap C
\end{equation}

\index{event network}
In general, when we deal with a number of events, they can be organized in a network, where the nodes represent the events and the 
edges represent their relational constraints. We refer to the network as an {\em event network}. It is assumed that the network maintains 
the complete information about the relational constraints between the events that are known at any point of time. When a new relational 
constraint is added, all its consequences are computed recursively in the network as a transitive closure of the new constraint and the 
existing ones, as shown in algorithm~\ref{algo:clock:add}. 

\begin{algorithm}[!htpb]
	\SetAlgoLined
	\SetKwProg{proc}{procedure}{}{end}
	\DontPrintSemicolon
	\proc{Add($R^{new}_{ij}$)}{
		add $R^{new}_{ij}$ to queue $Q$\; 
		\While {$Q$ is not empty}{
			get next $R^{new}_{ij}$ from $Q$\;
			$R^{old}_{ij} \leftarrow R_{ij}$\;
			$R_{ij} \leftarrow R_{ij} \cap R^{new}_{ij}$\;
			\If {$R_{ij} \subset R^{old}_{ij}$} {
				\For {each node $k$, such that Comparable$(k,j)$} {
					$R^{old}_{kj} \leftarrow R_{kj}$\;
					$R_{kj} \leftarrow R_{kj} \cap$ Constraint$(R_{ki},R_{ij})$\; 
					\If {$R_{kj} \subset R^{old}_{kj}$} {
						add $R_{kj}$ to queue $Q$\;
					} % EndIf
				} % EndFor
				\For {each node $k$, such that Comparable$(i,k)$} {
					$R^{old}_{ik} \leftarrow R_{ik}$\;
					$R_{ik} \leftarrow R_{ik} \cap$ Constraint$(R_{ij},R_{jk})$\; 
					\If {$R_{ik} \subset R^{old}_{ik}$} {
						add $R^n_{ik}$ to queue $Q$\;
					} % EndIf
				} % EndFor
			} % EndIf
		} % EndWhile
		\Return void\;
	} % EndProc
	\caption{Addition of a new constraint in event network}
	\label{algo:clock:add}
\end{algorithm}

A brief explanation of the algorithm is as follows. The superscripts $old$ and $new$ refer to the old (existing) and the new 
values of a constraint respectively. They are used as temporarily variables for comparison purposes, to decide if further recursion 
is needed. $Q$ maintains a queue of such new constraints, either externally specified, or generated at the intermediate stages of the 
recursive computation, for which the consequences are yet to be determined. The functions {\em Comparable$(i,k)$} and {\em Comparable$(k,j)$}
yield the set of all nodes $k$ in the network, for which the constraints $R_{ik}$ and $R_{kj}$ respectively are likely to be impacted by 
a change in $R_{ij}$. The function {\em Constraint$(\circ, \circ)$} is computed using equation~\ref{eqn:clock:implication}
%
Addition of a new constraint in the network restricts the existing constraints further. The updated constraint between a pair of 
events is the intersection of the existing constraints and the new ones resulting from the changes in other parts of the network.
If this intersection is an empty set, it implies the impossibility of the situation.
The recursive computation continues as long as there is something to update. Since there are 13 possible relations, the worst case 
complexity of the algorithm in a network with $N$ nodes is $13 \times \frac{(N-1)(N-2)}{2}$.

\paragraph{Example}

\noindent
Consider the statement: ``I clicked the switch to turn on the light''. It establishes a temporal constraint between two events 
clicking the switch, and the light being on. Designating the two events with \texttt{S} and \texttt{L}, the temporal constraint
between the events can be expressed as $R_{SL} \equiv$ \texttt{S} (m o) \texttt{L}. 
Now, consider that we add a statement ``Rajan was not there in the room when I clicked the switch''. We have introduced a new event
Rajan's presence in the room. Denoting the event with \texttt{R}, it's relation with clicking the switch can be formalized as 
$R_{RS} \equiv$ \texttt{R} (b m mi bi) \texttt{S}. 
The event network depicting the three events and the {\it specified temporal} constraints is shown in 
figure~\ref{fig:clock:Allen-algebra-example}(a).

\begin{figure}[htbp!]
        \subfigure[]{
                \begin{minipage}{\linewidth}
                \centerline{
			\epsfig{figure=./Clock/allen-algo-1.eps,width=0.5\linewidth}
                }
                \end{minipage}
        }
        \subfigure[]{
                \begin{minipage}{\linewidth}
                \centerline{
			\epsfig{figure=./Clock/allen-algo-2.eps,width=0.5\linewidth}
                }
                \end{minipage}
        }
        \subfigure[]{
                \begin{minipage}{\linewidth}
                \centerline{
			\epsfig{figure=./Clock/allen-algo-3.eps,width=0.5\linewidth}
                }
                \end{minipage}
        }
	\caption{Stages of event network update. (a) Constraints $R_{RS}$ and $R_{SL}$ asserted, (b) $R_{RL}$ computed
		from the assertions, (c) New constraint on $R_{RL}$ introduced.}
	\label{fig:clock:Allen-algebra-example}
\end{figure}

Addition of the second relation prompts a computation for the relation $R_{RL}$ between the events Rajan's presence in the room and 
the light being on. Prior to the computation, there is no constraint on the relation, i.e. it is a disjunction of all 13 of Allen's
relations. Following algorithm~\ref{algo:clock:add}, we find $R_{RL} = (\text{b si eq s d f oi mi bi})$. The updated network is shown 
in figure~\ref{fig:clock:Allen-algebra-example}(b). 

Let us assume that we get some further information, namely ``Rajan was there in the room at a later time while the light was 
on''. This new information can be formalized as an added constraint $R^{new}_{RL} \equiv (\text{oi si di})$. Inclusion
of the new constraint results in an update in the value of $R_{RL}$. The new value of $R_{RL}$ becomes (oi si), the intersection of the 
existing constraint and the new one. The update in $R_{RL}$ triggers a recursive update in the network. 
 
A change in $R_{RS}$ will be effected using the updated value of $R_{RL}$ and $R_{LS}$, the latter being a disjunction of 
the inverse of the relations in $R_{SL}$, i.e. $R_{LS} = (\text{mi oi})$. We can find $R^{new}_{RS} = (\text{bi mi oi})$. 
Taking its intersection with its existing value, we get the final value of $R_{RS}$ to be (bi mi). The recursion stops here, 
since there is no more nodes to traverse.  The updated network is shown in figure~\ref{fig:clock:Allen-algebra-example}(c). 
% 
This example illustrates the practical utility of Allen's interval algebra in reasoning with the temporal relations
of interval events. The reader is advised to verify the computations in this example and the semantic consistency of the 
results with the stated facts.

\subsection{Conceptual neighborhood}

The discussions in the earlier sections of this chapter show that despite various clock synchronization methods, the sequence of 
point events, recorded by different processors on a distributed system, cannot always be precisely determined. For example, if the 
audio of a gun-shot and the visual scene of a man falling are recorded by a microphone and a video camera connected to different 
processors in a distributed system, it may be difficult to establish the exact sequence of the start and the end points of the 
audio and the visual events. There are many factors other than network delays in a distributed system, which also contributes to 
the uncertainty, for example delays introduced by the recording equipment, the difference in the speeds of light and sound, 
and inherent nondeterminism in start and end points of an event~\footnote{For example, when does a man, who is already in motion 
(say walking), start falling? when does he stop falling?}. Moreover, in the last section, we have seen that the inferred relation 
between event pairs cannot always be unambiguously resolved. Thus, there are many situations when we may have to be satisfied with 
the inexact knowledge about the relations between interval events.

  
\begin{figure}[htbp!]
	\centerline{
		\epsfig{figure=./Clock/allen-semantics.eps,width=0.8\linewidth}
	}
	\caption{Semantics of Allen's relations}
	\label{fig:clock:Allen-semantics}
\end{figure}
 
\index{inexact temporal relations}
The ambiguity in establishing the temporal relations between the events motivates exploration of the nature of the inexactness
in the relations. Allen's relations can be placed on a on a two dimensional grid~\citep{Freska:1992} as shown in
figure~\ref{fig:clock:Allen-semantics}, to understand their semantics. The edges of the grid are marked with the comparison between 
the start and the end points of the two events, designated as \texttt{A} and \texttt{B}. The 13 relations are depicted as cells in 
the grid in their respective positions for the comparison results. We observe that only one of the comparison parameters change 
between any two adjacent relations, vertically, horizontally, or diagonally.
It implies that it is possible to move from one relation to any of its adjacent relation by continuously deforming (i.e. by 
continuously changing the end-points of) either or both the events without satisfying any other relation at an intermediate stage. 
Thus, a small error in perceiving the end-points of the events is likely to result in misclassification of their relation to 
an adjacent one.

\index{conceptual neighbors}
\index{conceptual neighborhood}
The adjacent relations in figure~\ref{fig:clock:Allen-semantics} are called {\em conceptual neighbors}. A set of relations forms 
a {\em conceptual neighborhood}, if the its members are connected through conceptual neighborhood relations (e.g. {\bf b}, {\bf m} and 
{\bf o}).  
Thus, when the relation between two events are not known with certainty, it is still possible to represent their relation in an 
inexact fashion as a conceptual neighborhood. Such inexact descriptions are often useful for logical reasoning with interval events, 
as illustrated in an earlier example.
An alternate way to deal with such inexact description is to have fuzzy representation of the event boundaries, and define fuzzy
membership functions for the Allen's relations~\citep{Wattamwar:2008}.

\subsection{Spatial Events}

\index{multi-dimensional events}
The events dealt with a distributed system need not always be restricted to temporal dimension alone. For example, visual events
as captured with video-cameras cover some spatial extents as well. Allen's interval algebra and its extensions in the forms of
bit-encoded or fuzzy representations, are equally applicable to any spatial dimension. The semantics of the relations need to be 
appropriately modified in such cases, for example, the relation b (before) may stand for ``to the left of'' or ``below'', considering
that the space is measured from left to right, or from bottom to top. 
%
As in the case of temporal dimension, there may be ambiguities in ascertaining the spatial relation between the end-points of
the events, as exemplified in the differences found in relative positions of objects seen with two different cameras, separated
in space. This necessitates the need for considering conceptual neighborhoods and approximate queries with spatial events as well.

It may seem to be logical to represent relations between multi-dimensional events as tuples of elementary relations, each of which 
refers to the relation of the projections of the events on a specific dimension. For example, a visual event in a video is restricted 
to two dimensional space ($x,y$) and time  ($t$), and one may be tempted to represent the relation between such events as 
($R^X, R^Y, R^T$), where each of the $R$s can be one of the Allen's relation. But, it must  be appreciated that all discussions on 
interval events presented in the earlier sections are valid so long there is a strict ordering relation between the event endpoints. 
With two or more spatio-temporal dimensions, the orderings between the event endpoints become partial, and ambiguities creep in. For 
example, it is easy to see that the two distinct relations presented in figures~\ref{fig:clock:Allen-ambiguity}(a) and (b) cannot 
be distinguished with Allen's relations in the projected spaces alone.


\begin{figure}[htbp!]
	\subfigure[\texttt{B} is contained in \texttt{A}]{
                \begin{minipage}{0.5\linewidth}
                \centerline{
			\epsfig{figure=./Clock/allen-ambiguity-1.eps,width=0.8\linewidth}
                }
                \end{minipage}
        }
	\subfigure[\texttt{B} is outside \texttt{A}]{
                \begin{minipage}{0.5\linewidth}
                \centerline{
			\epsfig{figure=./Clock/allen-ambiguity-2.eps,width=0.8\linewidth}
                }
                \end{minipage}
        }
	\caption{Ambiguity in Allen's relations when extended to multi-dimensional space}
	\label{fig:clock:Allen-ambiguity}
\end{figure}

\index{containment relations}
The ambiguity can be resolved with introduction of one more set of relations that specifies the of intersection between two 
events~\citep{Wattamwar:2008}. These containment relations are shown in figure~\ref{fig:clock:containment}. These relations
can also have bit-string or fuzzy encodings as the Allen's relation. Thus, the relation between two events in a multi-dimensional 
space of dimension $N$ can be specified with an $N+1$-tuple, where one of the members represent the relation in containment 
space and each of the other $N$ members specify an Allen's relation for the  projections of the events on one of the dimensions.

\begin{figure}[htbp!]
	\centerline{
		\epsfig{figure=./Clock/containment.eps,width=0.8\linewidth}
	}
	\caption{Containment relations}
	\label{fig:clock:containment}
\end{figure}
